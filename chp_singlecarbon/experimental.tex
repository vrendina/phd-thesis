%%% Model substrates
\def \CMPxbaa{trimethyl(6-methyl-6-phenylcyclohex-1-enyloxy)silane}		% major regioisomer tms ether
\def \CMPxbab{trimethyl(3-methyl-3-phenylcyclohex-1-enyloxy)silane}		% minor regioisomer silyl
\def \CMPxbac{2-methyl-2-phenyl-cyclohexanone}		% major regioisomer methyl phenyl
\def \CMPxbad{3-methyl-3-phenyl-cyclohexanone}		% minor regioisomer methyl phenyl 
\def \CMPxbae{homologated estrone 3-methyl ether major}		% estrone homologated major
\def \CMPxbaf{homologated estrone 3-methyl ether minor}		% estrone homologated minor

%%% Pre-reductive alkylation
\def \CMPxbag{Hajos-Parrish ketone}		% Hajos-Parrish ketone
\def \CMPxbah{Hajos-Parrish keto-alcohol}		% Reduced Hajos-Parrish ketone
\def \CMPxbai{($\pm$)-\textit{d}$_1$-Hajos-Parrish	keto-alcohol}		% 1D reduction

%%% Generation 1 compounds
\def \CMPxbaj{(3,5-dimethoxyphenyl)methanol}		% electrophile step 1
\def \CMPxbak{(2-chloro-3,5-dimethoxyphenyl)methanol}		% electrophile step 2, chlorination
\def \CMPxbal{1-(bromomethyl)-2-chloro-3,5-dimethoxybenzene}		% electrophile step 3, bromination
\def \CMPxbam{2-chloro-1-(iodomethyl)-3,5-dimethoxybenzene}		% electrophile step 4, iodide
\def \CMPxban{($-$)-keto-alcohol}		% reductive alkylation product
\def \CMPxbao{($\pm$)-exocyclic ene-ol}		% wittig without protection, olefin in wrong place
\def \CMPxbap{($-$)-keto-\textit{tert}-butyldimethylsilyl ether}		% tbs protection
\def \CMPxbaq{($+$)-\textit{tert}-butyldimethylsilyl	ether-alkene}		% wittig
\def \CMPxbar{($-$)-trisubstituted ene-ol}		% isomerization & deprotection
\def \CMPxbas{($-$)-trisubstituted	ene-one}		% oxidation
\def \CMPxbat{($\pm$)-$\beta$-methyl ketone}		% hydrogenation
\def \CMPxbau{($\pm$)-$\alpha$-methyl ketone}		% hydrogenation
\def \CMPxbav{($+$)-ene-decalone}		% homologation of trisub olefin


%%% Generation 2 compounds (benzyl protecting group)
\def \CMPxbaw{(2-(benzyloxy)-6-chloro-3-methoxyphenyl)methanol}		% electrophile 1, chlorination
\def \CMPxbax{2-(benzyloxy)-3-(bromomethyl)-4-chloro-1-methoxybenzene}		% electrophile 2,
% bromination
\def \CMPxbay{2-(benzyloxy)-4-chloro-3-(iodomethyl)-1-methoxybenzene}		% electrophile 3, iodide
\def \CMPxbaz{($-$)-keto-alcohol}		% reductive alkylation product
\def \CMPxbba{($\pm$)-decahydrocyclopenta[\textit{a}]xanthene}		% wittig without protection gen 2
\def \CMPxbbb{($-$)-keto-\textit{tert}-butyldimethylsilyl ether}		% TBS protection
\def \CMPxbbc{($+$)-tert-butyldimethylsilyl ether-alkene}		% wittig
\def \CMPxbbd{($+$)-1,1-disubstituted	ene-ol}		% deprotection
\def \CMPxbbe{($+$)-1,1-disubstituted ene-one}		% oxidation
\def \CMPxbbf{($-$)-trisubstituted ene-ol}		% isomerization and deprotection
\def \CMPxbbg{($-$)-trisubstituted ene-one}		% oxidation of isomerized olefin
\def \CMPxbbh{($+$)-1,1-disubstituted ene-decalone major}		% homologation exocyclic major
\def \CMPxbbi{($+$)-1,1-disubstituted ene-decalone minor}		% homologation exocyclic minor
\def \CMPxbbj{($-$)-trisubstituted ene-decalone}		% homologation major trisub olefin

%%% cross-over experiment compounds
\def \CMPxbbk{2-(benzyloxy)-3-methoxybenzoic acid}		%
\def \CMPxbbl{\textit{d}$_2$-(2-(benzyloxy)-6-chloro-3-methoxyphenyl)methanol}		%
\def \CMPxbbm{\textit{d}$_2$-2-(benzyloxy)-3-(bromomethyl)-4-chloro-1-methoxybenzene}		% bromination
\def \CMPxbbn{\textit{d}$_2$-2-(benzyloxy)-4-chloro-3-(iodomethyl)-1-methoxybenzene}		% iodide
\def \CMPxbbo{($\pm$)-\textit{d}$_3$-keto-alcohol}		% reductive alkylation product 3D
      % For compound systematic names



\subsection{General Information}

\subsubsection{General Procedures}
Unless stated otherwise, all reactions were carried out in flame-dried glassware under an atmosphere
of argon passed through a tower of finely powdered Drierite$\textsuperscript{\textregistered}$ in dry, degassed solvent with standard
Schlenk or vacuum-line techniques. Particularly air-sensitive manipulations were performed in an
MBraun Unilab nitrogen atmosphere glove box. Flash column chromatography, driven by compressed air,
was performed according to the procedure of Still \textit{et al.}\footnote{{\frenchspacing Still, W. C.; Kahn, M.; Mitra,
A. Rapid Chromatographic Technique for Preparative Separations with Moderate Resolution. \textit{J.
Org. Chem.} \textbf{1978}, \textit{43}, 2923-2925.}} with
ZEOPrep 60 Eco 40-63 $\mu$m silica gel. Analytical thin-layer chromatography (TLC) was performed using 0.25 mm
silica gel 60 F254 plates purchased from EMD Chemicals. TLC plates were visualized by exposure to ultraviolet light and/or exposure to ceric
ammonium molybdate, \textit{p}-anisaldehyde, or potassium permanganate stains. 

\subsubsection{Materials}
Tetrahydrofuran (THF), dichloromethane (\ce{CH2Cl2}), diethyl ether (Et$_2$O), benzene,
acetonitrile (CH$_3$CN), and \textit{N},\textit{N}-dimethylformamide (DMF) were dispensed under UHP
argon from a Glass Contour solvent purification system custom manufactured by SG Waters, LLC
(Nashua, NC). Pyridine, phosphorus tribromide (PBr$_3$), \textit{N}-chlorosuccinimide (NCS), sodium
iodide (NaI), methyltriphenylphosphonium bromide (Ph$_3$PCH$_3$Br), boron trifluoride etherate
(\ce{BF3.OEt2}), dimethyl sulfoxide (DMSO), trimethylaluminum (AlMe$_3$), methanol,
\textit{tert}-butyldimethylsilyl triuoromethanesulfonate (TBSOTf), tert-butyldimethylsilyl chloride
(TBSCl), triethylamine (Et$_3$N), imidazole, D-phenylalanine (D-Phe), pyridinium p-toluenesulfonate
(PPTS), deuterochloroform (CDCl$_3$), carbon tetrachloride (CCl$_4$), 1,3-dichloro-5,5-dimethylhydantoin, and acetone
were purified and dried according to the reported procedures.\footnote{{\frenchspacing Armarego, W. L. F.; Chai, C. L. \textit{Purification of
Laboratory Chemicals}, 5th ed.; Butterworth-Heinemann: Oxford, 2003.}} Estrone 3-methyl ether,
potassium carbonate (K$_2$CO$_3$), 3,5- dimethoxybenzoic acid, \textit{n}-butyllithium (\textit{n}-BuLi) in hexanes, sodium borohydride
(NaBH$_4$), lithium aluminum hydride (LiAlH$_4$), lithium aluminum deuteride (LiAlD$_4$), sodium
borodeuteride (NaBD$_4$),
sodium hydride (NaH), ethanol (EtOH), chloroform (CHCl$_3$), sodium chlorite
(NaClO$_2$), pyridinium chlorochromate (PCC), platinum(IV)oxide (PtO$_2$), 10\% wt/wt palladium on
carbon (Pd/C), lithium wire, sodium chunks, tetra-\textit{n}-butylammonium fluoride
hydrate (\ce{TBAF.xH2O}), hydrogen peroxide in water (30\% wt/wt), and
Celite$\textsuperscript{\textregistered}$ 545 were purchased from Sigma-Aldrich and used without further purification. Sodium
chloride (NaCl), ammonium chloride (NH$_4$Cl), sodium bicarbonate (NaHCO$_3$), potassium
carbonate (K$_2$CO$_3$), sodium hydroxide (NaOH), sodium sulfate (Na$_2$SO$_4$), sodium thiosulfate
(\ce{Na2S2O3}), sodium dihydrogen phosphate (\ce{NaH2PO4}), and magnesium sulfate (MgSO$_4$) were
purchased from Fisher Scientific and used without further purification.
Methyltriphenylphosphonium iodide was prepared from triphenylphosphine (Aldrich), and
methyl iodide (Aldrich) by stirring in benzene for 2 hours, filtering, washing with hexanes, and
drying over \ce{P2O5} before use. Molecular sieves (3\AA, 4-8 mesh) were purchased from Aldrich and
activated by drying under vacuum (approx. 30 mm Hg) at 250 \degc\  for at least 6 hours prior to
use.
Rhodium chloride hydrate (\ce{RhCl3.H2O}) was purchased from Pressure Chemical Co. and used
without further purification. Anhydrous ammonia was purchased from Airgas Inc. and distilled
from sodium metal prior to use. Dess-Martin Periodinane was prepared according to the reported
literature procedure.\footnote{{\frenchspacing Meyer, S. D.; Schreiber, S. L. Acceleration of the
Dess-Martin Oxidation by Water. \textit{J. Org. Chem.} \textbf{1994}, \textit{59}, 7549-7552.}} Scandium triflate (Sc(OTf)$_3$) (99\%) was purchased from Sigma-Aldrich, finely powdered, and then dried at 200 \degc\  over \ce{P2O5} for 24 hours under high vacuum (0.1 mm Hg). The dried scandium triflate was taken into a dry box using rigorous Schlenk techniques.\footnote{For information on handling scandium triflate, refer to
 the previous chapter.} Trimethylsilyldiazomethane (TMSD) and
 phenyldimethylsilyldiazomethane (PDMSD) were prepared according to the reported literature
 procedure\crossref{ref:bshioriorgsynth} and were stored over 3\AA\  molecular sieves at $-$40
 \degc\ in a drybox.
 Note:
 TMSD is both non-explosive and non-mutagenic, however it is extremely toxic and should be handled with the appropriate precautions.

\subsubsection{Instrumentation}
Infrared spectra were recorded on a Bruker Alpha-p spectrometer. Bands are reported as
strong (s), medium (m), weak (w), broad strong (bs), broad medium (bm), and broad weak (bw).
Optical rotation data were recorded on a Rudolph research Autopol IV automatic polarimeter and
is reported as the average of five readings. Melting points were recorded on a Digimelt MPA160
SRS and are uncorrected. Sonication was performed with a Misonix$\textsuperscript{\textregistered}$ Sonicator 3000 equipped
with a Laude external circulator. $^1$H NMR spectra were recorded on a Varian VNMRS (500
MHz), INOVA (500 MHz), or VNMRS (400 MHz) spectrometer. Chemical shifts are reported in
ppm from tetramethylsilane with the solvent resonance as the internal standard (CHCl$_3$: $\delta$
7.26).
Data are reported as follows: chemical shift, multiplicity (s = singlet, d = doublet, dd = doublet of
doublets, ddd = doublet of doublet of doublets, dddd = doublet of doublet of doublet of doublets, t
= triplet, m = multiplet), coupling constants (Hz), and integration. $^{13}$C NMR spectra were
recorded on a Varian VNMRS (125 MHz), INOVA (125 MHz), or VNMRS (100 MHz) spectrometer with complete
proton decoupling. Chemical shifts are reported in ppm from tetramethylsilane with the solvent as
the internal reference (CDCl$_3$: $\delta$ 77.16, DMSO-\textit{d}$_6$: $\delta$ 39.52).
High-resolution mass spectra were obtained at the Boston College Mass Spectrometry Facility.
Supercritical fluid chromatography (SFC) data were obtained on a Berger Instruments system using a
Daicel CHIRALPAK AS-H column ($\phi$ 4.6 mm, 25 cm length). Gas chromatography (GC) analysis was performed on an Agilent Technologies 7890A system equipped with a flame ionization detector and HP-5
column (30 m x 0.320 mm x 0.25 $\mu$m).


\pagebreak
%%%%%%%%%%%%%%%%%%%%%%%%%%%%%%%%%%%%%%%%%%%%%%%%%%%%%%%%%%%%%%%%%%%%%%%%%%%%%%%%%%%%%%%%
% Begin experimental procedures for single carbon homologation chapter.
%%%%%%%%%%%%%%%%%%%%%%%%%%%%%%%%%%%%%%%%%%%%%%%%%%%%%%%%%%%%%%%%%%%%%%%%%%%%%%%%%%%%%%%%
\subsection{Experimental Procedures and Characterization Data}     
%% \noindent cannot be present directly under section headings
%***************[xbaa]%***************%
\begin{wrapfigure}{l}{0.95in}
  \vspace{-25pt}
  \begin{center}
    \includegraphics[scale=0.8]{chp_singlecarbon/images/xbaa}
    \begin{textblock}{1}(1.6,-0.7) \cmp{xbaa} \end{textblock}
  \end{center}
  \vspace{-30pt}
\end{wrapfigure}
\textbf{\CMPxbaa}\ (\ref{cmp:xbaa}). In a drybox, Yb(OTf)$_3$ (19.2 mg, 0.0310 mmol, 0.100 equiv)
was weighed directly into a 1.5 mL vial. A solution of ketone \ref{cmp:xaab} (54.0 mg, 0.310 mmol,
1.0 equiv) in 1.55 mL of \ce{CH2Cl2} was then transferred directly to the solid Yb(OTf)$_3$. TMSD
(251 $\mu$L, 0.630 mmol, 2.00 equiv, 2.47 M in hexanes) was introduced dropwise, and the reaction
mixture was allowed to stir for 27 hours in the drybox. The vessel was then removed from the drybox, and the reaction mixture was
poured into saturated aqueous NaHCO$_3$ (20 mL). The product was extracted with Et$_2$O (3 x 10 mL),
and the combined organics were washed with saturated aqueous NaCl (20 mL), dried over \ce{Na2SO4},
filtered, and concentrated. Purification by column chromatography (100\% hexanes) afforded the
desired enol silane \ref{cmp:xbaa} as a colorless oil (61.8 mg, 76.5\%). \\
R$_f$ = 0.35 (100\% hexanes); $^1$H NMR (CDCl$_3$, 500 MHz) $\delta$ 7.38-7.35 (m, 2H), 7.30-7.26
(m, 2H), 7.19-7.14 (m, 1H), 4.93 (dd, \textit{J} = 3.9, 3.9 Hz, 1H), 2.12-2.07 (m, 2H), 1.88 (ddd,
\textit{J} = 13.2, 6.6, 2.9 Hz, 1H), 1.71 (ddd, \textit{J} = 13.2, 11.3, 2.9 Hz, 1H), 1.49-1.42 (m,
1H), 1.45 (s, 3H), 1.38-1.27 (m, 1H), 0.12 (s, 9H); $^{13}$C NMR (CDCl$_3$, 125 MHz) $\delta$
154.39, 148.23, 127.81, 127.14, 125.58, 103.74, 43.96, 41.17, 26.20, 24.74, 19.08, 0.51; IR (neat)
2961 (bm), 2932 (bm), 2838 (w), 1657 (m), 1248 (s),1182 (s), 1152 (w), 843 (s), 759 (m), 698 (m)
cm$^{-1}$; HRMS (ESI+) Calcd. for \ce{C16H25OSi} [M+H]$^+$: 261.1675; Found 261.1671.
% ***************[xbaa]%***************%

\vspace{10pt}
%***************[xbab]%***************%
\begin{wrapfigure}{l}{1.15in}
  \vspace{-25pt}
  \begin{center}
    \includegraphics[scale=0.8]{chp_singlecarbon/images/xbab}
    \begin{textblock}{1}(0.25,-0.8) \cmp{xbab} \end{textblock}
  \end{center}
  \vspace{-30pt}
\end{wrapfigure}\noindent \textbf{\CMPxbab}\ (\ref{cmp:xbab}). Authentic material for comparison
purposes was prepared according to the literature procedure.\footnote{{\frenchspacing Posner, G. H.;
Lentz, C. M.	\textit{J. Am. Chem. Soc.} \textbf{1979}, \textit{101}, 934-946.}}\\
R$_f$ = 0.38 (3\% ethyl acetate in hexanes v/v); $^1$H NMR (CDCl$_3$, 500 MHz) $\delta$7.40-7.37 (m,
2H), 7.31-7.27 (m, 2H), 7.19-7.15 (m, 1H), 4.95-4.94 (m, 1H), 2.08-1.98 (m, 2H), 1.83-1.77 (m, 1H),
1.66-1.54 (m, 2H), 1.48-1.41 (m, 1H), 1.40 (s, 3H), 0.25 (s, 9H); $^{13}$C NMR (CDCl$_3$, 125 MHz)
$\delta$ 150.68, 150.43, 128.02, 126.79, 125.65, 113.50, 40.12, 39.05, 30.33, 29.96, 19.72, 0.63; IR
(neat) 2958 (bm), 2933 (bm), 1661 (m), 1251 (m), 1196 (s) 894 (m), 843 (s), 760 (m), 699 (m)
cm$^{-1}$; HRMS (ESI+) Calcd. for \ce{C16H25OSi} [M+H]$^+$: 261.1675; Found 261.1662.

%***************[xbab]%***************%

\vspace{10pt}
%***************[xbac]%***************%
\begin{wrapfigure}{l}{0.85in}
  \vspace{-25pt}
  \begin{center}
    \includegraphics[scale=0.8]{chp_singlecarbon/images/xbac}
  \end{center}
  \vspace{-30pt}
\end{wrapfigure}\noindent \textbf{\CMPxbac}\ (\ref{cmp:xbac}). To a solution of silyl enol ether
\ref{cmp:xbaa} (57.1 mg, 0.219 mmol, 1.00 equiv) in 1.1 mL of THF, TBAF (1.0 mL, 1.0 mmol, 4.8
equiv, 1.0 M solution in THF) was added. After 40 minutes at 23 \degc, the reaction mixture was
poured into H$_2$O (20 mL). The product was extracted with ethyl acetate (3 x 15 mL), and the
combined organics were washed with saturated aqueous NaCl (20 mL), dried over \ce{Na2SO4}, filtered,
and concentrated. The crude residue was then passed through a plug of silica gel eluting with ethyl acetate and then
concentrated to afford the desired product \ref{cmp:xbac} as a yellow oil (46.2 mg, quantitative).\\
R$_f$ = 0.48 (10\% ethyl acetate in hexanes v/v); $^1$H NMR (CDCl$_3$, 500 MHz) $\delta$ 7.37-7.33
(m, 2H), 7.26- 7.22 (m, 1H), 7.24 (tt, \textit{J} = 6.8, 1.2 Hz, 1H), 7.20-7.17 (m, 2H), 2.72-2.66 (m,
1H), 2.42-2.28 (m,2H), 2.00-1.93 (m, 1H), 1.78-1.67 (m, 4H), 1.27 (s, 3H); $^{13}$C NMR (CDCl$_3$,
125 MHz) $\delta$ 214. 17, 143.42, 129.11, 126.69, 126.22, 54.51, 40.05, 38.30, 28.59, 22.00; IR
(neat) 2934 (bm), 2863 (bm), 1708 (s), 1495 (w), 1448 (w), 759 (w), 702 (m), 551 (m) cm$^{-1}$; HRMS
(ESI+) Calcd. for \ce{C13H17O} [M+H]$^+$: 189.1279; Found 189.1284.
% ***************[xbac]%***************%

\vspace{10pt}
% ***************[xbad]%***************%
\begin{wrapfigure}{l}{1.0in}
  \vspace{-25pt}
  \begin{center}
    \includegraphics[scale=0.8]{chp_singlecarbon/images/xbad}
  \end{center}
  \vspace{-30pt}
\end{wrapfigure}\noindent \textbf{\CMPxbad}\ (\ref{cmp:xbad}). To a solution of silyl enol ether
\ref{cmp:xbab} (46.3 mg, 0.178 mmol, 1.00 equiv) in 0.9 mL of THF, TBAF (0.40 mL, 0.43 mmol, 2.4
equiv, 1.0 M solution in THF) was added. After 40 minutes at 23 \degc, the reaction mixture was
poured into H$_2$O (20 mL). The product was extracted with ethyl acetate (3 x 15 mL), and the
combined organics were washed with saturated aqueous NaCl (20 mL), dried over \ce{Na2SO4}, filtered,
and concentrated. The crude residue was then passed through a plug of silica gel eluting with ethyl acetate and then
concentrated to afford the desired product \ref{cmp:xbad} as a yellow oil. \\
R$_f$ = 0.30 (10\% ethyl acetate in hexanes v/v); $^1$H NMR (CDCl$_3$, 500 MHz) $\delta$ 7.35-7.31
(m, 4H), 7.23- 7.18 (m, 1H), 2.88 (d, \textit{J} = 14.2 Hz, 1H), 2.44 (d, \textit{J} = 14.4 Hz, 1H),
2.34-2.29 (m, 2H), 2.22-2.16 (m, 1H), 1.96-1.84 (m, 2H), 1.72-1.63 (m, 1H), 1.33 (s, 3H); $^{13}$C
NMR (CDCl$_3$, 125 MHz) $\delta$ 211.58, 147.57, 128.65, 126.31, 125.70, 53.21, 42.94, 40.92, 38.07,
29.90, 22.14; IR (neat) 2957 (bm), 2872 (bm), 1710 (s), 1498 (w), 1422 (m), 1228 (m), 1031 (bw), 764
(m), 700 (s) cm$^{-1}$; HRMS (ESI+) Calcd. for \ce{C13H17O} [M+H]$^+$: 189.1279; Found 189.1279.
% ***************[xbad]%***************%

\vspace{10pt}
%***************[xbae]%***************%
\begin{wrapfigure}{l}{1.6in}
  \vspace{-28pt}
  \begin{center}
    \includegraphics[scale=0.8]{chp_singlecarbon/images/xbae}
  \end{center}
  \vspace{-30pt}
\end{wrapfigure}\noindent \textbf{\CMPxbae}\ (\ref{cmp:xbae}). In a drybox, Sc(OTf)$_3$ (3.7 mg,
0.0075 mmol, 0.050 equiv) was weighed directly into a 1.5 mL vial equipped with a magnetic stirbar.
A solution of estrone 3-methyl ether (42.6 mg, 0.150 mmol, 1.00 equiv) in CDCl$_3$ (0.53 mL) was
transferred directly to the solid Sc(OTf)$_3$. The cloudy gray suspension was stirred for 15 minutes
at which point TMSD (121 $\mu$L, 0.300 mmol, 2.00 equiv, 2.47 M in hexanes) was introduced dropwise.
The entire reaction mixture (including any residual solids) was transferred via glass pipette to a J. Young NMR tube,
and the vial was rinsed with an additional 0.2 mL of CDCl$_3$. The reaction tube was removed from
the drybox, connected to a nitrogen manifold, and allowed to stand for 24 hours at 23 \degc.
1,3,5-trimethoxybenzene (11.0 mg, 0.654 mmol) was added, and $^1$H NMR analysis indicated a
72\% yield of the major enol silane. The reaction mixture was poured into H$_2$O (5 mL), and the
product was extracted with \ce{CH2Cl2} (3 x 10 mL). The combined organics dried over \ce{Na2SO4},
filtered, and concentrated. The crude residue was then dissolved in 1 mL of THF, TBA\ce{F.x}H$_2$O
(168 mg, excess) was added as a solid, and the reaction mixture was allowed to stir for 30 minutes
at 23 \degc. The reaction mixture was then poured into H$_2$O (5 mL) and the product was extracted
with Et$_2$O (3 x 5 mL), and the combined organics were passed through a plug of silica gel rinsing
with ethyl acetate (10 mL) and concentrated. Purification by column chromatography (15\% ethyl
acetate in hexanes v/v) afforded the desired homologated estrone derivative \ref{cmp:xbae} as a
white solid (30.4 mg, 67.9\%), mp 136-138 \degc.\\
R$_f$ = 0.30 (15\% ethyl acetate in hexanes v/v); $^1$H NMR (CDCl$_3$, 500 MHz) $\delta$ 7.22 (dd,
\textit{J} = 8.8, 0.5 Hz, 1H), 6.72 (dd, \textit{J} = 8.9, 2.9 Hz, 1H), 6.63 (d, 2.9 Hz, 1H), 3.78
(s, 3H), 2.88-2.83 (m, 2H), 2.67 (ddd, \textit{J} = 14.2, 14.2, 6.8 Hz, 1H), 2.38 (dddd, 11.5, 4.2,
4.2, 4.2 Hz, 1H), 2.28-2.21 (m, 2H), 2.16-2.05 (m, 2H), 1.99-1.93 (m, 1H), 1.89 (ddd, \textit{J} =
13.9, 3.4, 3.4 Hz, 1H), 1.73 (ddd, \textit{J} = 13.7, 13.7, 3.9 Hz, 1H), 1.69-1.58 (m, 1H),
1.55-1.39 (m, 4H), 1.34-1.25 (m, 1H), 1.13 (s, 3H); $^{13}$C NMR (CDCl$_3$, 125 MHz) $\delta$
216.45, 157.69, 137.76, 132.60, 126.48, 113.59, 111.77, 55.33, 50.44, 48.54, 43.17, 38.99, 37.32,
32.66, 30.24, 26.78, 26.07, 26.03, 23.08, 17.02; IR (neat) 2930 (bs), 2863 (bm), 1703 (s), 1610 (w),
1502 (m), 1429 (bm), 1254 (m), 1237 (m), 1040 (w) cm$^{-1}$; HRMS (ESI+) Calcd. for \ce{C20H27O2}
[M+H]$^+$: 299.2011; Found 299.1999.
%***************[xbae]%***************%

\vspace{10pt}
%***************[xbaf]%***************%
\begin{wrapfigure}{l}{1.7in}
  \vspace{-25pt}
  \begin{center}
    \includegraphics[scale=0.8]{chp_singlecarbon/images/xbaf}
  \end{center}
  \vspace{-30pt}
\end{wrapfigure}\noindent \textbf{\CMPxbaf}\ (\ref{cmp:xbaf}). Isolated as the minor regioisomer in
the procedure for compound \ref{cmp:xbae}. Purification by column chromatography (15\% ethyl acetate
 in hexanes v/v) afforded the minor regioisomer as a white solid (9.9 mg, 22\%), mp 176-180 \degc.\\
 R$_f$ = 0.17 (15\% ethyl acetate in hexanes v/v); $^1$H NMR (CDCl$_3$, 500 MHz) $\delta$ 7.22 (d
 \textit{J} = 8.3 Hz, 1H), 6.73 (dd, \textit{J} = 8.8, 2.9 Hz, 1H), 6.64 (d, \textit{J} = 2.9 Hz,
 1H), 3.78 (s, 3H), 2.89-2.83 (m, 2H), 2.47-2.21 (m, 5H), 2.23 (d, \textit{J} = 13.7 Hz, 1H),
 2.16-2.09 (m, 1H), 2.14 (d, \textit{J} = 13.4, 2.4 Hz, 1H), 1.67-1.42 (m, 5H), 1.41-1.24 (m, 2H),
 0.83 (s, 3H); $^{13}$C NMR (CDCl$_3$, 125 MHz) $\delta$ 211.83, 157.74, 137.95, 132.58, 126.45,
 113.64, 111.84, 56.93, 55.38, 48.12, 43.72, 41.38, 41.33, 39.66, 38.38, 30.20, 26.76, 26.50, 25.72,
 17.88; IR (neat) 2922 (bs), 2861 (bm), 1709 (s), 1612 (w), 1501 (m), 1256 (s), 1038 (m), 810 (w),
 79 (w) cm$^{-1}$; HRMS (ESI+) Calcd. for \ce{C20H27O2} [M+H]$^+$: 299.2011; Found 299.2015.
%***************[xbaf]%***************%

%***************[xbag]%***************%
\begin{wrapfigure}{l}{1.0in}
  \vspace{-15pt}
  \begin{center}
    \includegraphics[scale=0.8]{chp_singlecarbon/images/xbag}
    \begin{textblock}{1}(2,-0.7) \cmp{xbag} \end{textblock}
  \end{center}
  \vspace{-30pt}
\end{wrapfigure}\noindent \textbf{\CMPxbag}\ (\ref{cmp:xbag}). A 40 mL vial (95 mm x 25 mm) equipped
with a magnetic stirbar and a rubber septum was charged with 2-methyl-2-(3-
oxopentyl)cyclopentane-1,3-dione\footnote{{\frenchspacing Hajos, Z. G.; Parrish, D. R.	
  \textit{Org. Synth.} \textbf{1990},	\textit{Coll. Vol. 7}, 363.}} (2.00 g, 10.2 mmol, 1.00 equiv),
  D-Phe (505 mg, 3.06 mmol, 0.300 equiv), and PPTS (1.28 g, 5.09 mmol, 0.499 equiv). DMSO
(0.73 mL) was added with a syringe, and the resulting suspension was stirred for 5 minutes at
room temperature. The vial was then sonicated (60 W) continuously at 50 \degc\  for 24 hours. 20
minutes into the reaction period at 50 \degc, the reaction mixture was observed to be dark yellow
and homogeneous. The crude reaction mixture was directly loaded onto a flash column and eluted
with 50\% Et$_2$O in pentane (v/v) to afford the desired product \ref{cmp:xbag} as a colorless oil
(1.61 g, 88.6\%) with 91\% ee (AS-H, 50 \degc, 150 psi, 1.0 mL/min, 3\% MeOH, $\lambda$ = 220 nm;
t$_R$ = 10.06 min (minor), 10.80 min (major)). \\
R$_f$ = 0.50 (60\% Et$_2$O in pentane v/v); $^1$H NMR (CDCl$_3$, 500 MHz) $\delta$ 2.96-2.87 (m,
1H), 2.85-2.73 (m, 2H), 2.60-2.37 (m, 3H), 2.07 (ddd, \textit{J} = 13.4, 5.1, 2.2 Hz, 1H), 1.85 (ddd, \textit{J} = 13.9, 13.9, 5.9 Hz,
1H), 1.78 (d, \textit{J} = 1.2 Hz, 3H), 1.29 (s, 3H); $^{13}$C NMR (CDCl$_3$, 100 MHz) $\delta$
217.74, 197.99, 162.55, 129.95, 48.99, 35.54, 32.92, 28.94, 24.60, 21.38, 10.89; HRMS (ESI+) Calcd. for \ce{C11H15O2} [M+H]$^+$: 179.1072; Found 179.1076.
\begin{figure}[h]
\vspace{10pt}
\centering
\includegraphics[width=5.25in]{chp_singlecarbon/images/sfc/xbag_sfc.jpg}
\caption{SFC trace for \CMPxbag~(\ref{cmp:xbag})}
\vspace{-10pt}
\end{figure}
%***************[xbag]%***************%

\pagebreak
%***************[xbah]%***************%
\begin{wrapfigure}{l}{1.0in}
  \vspace{-12pt}
  \begin{center}
    \includegraphics[scale=0.8]{chp_singlecarbon/images/xbah}
  \end{center}
  \vspace{-30pt}
\end{wrapfigure}\noindent \textbf{\CMPxbah}\ (\ref{cmp:xbah}). Hajos-Parrish ketone \ref{cmp:xbag}
(3.49 g, 19.6 mmol, 1.00 equiv) was dissolved in 70 mL of EtOH, and the resulting
homogeneous solution was cooled to $-$25 \degc. Sodium borohydride (0.233 g, 6.16
mmol, 0.314 equiv) was added as a solid, and the mixture was closely monitored
by TLC. After 20 minutes, the reaction was judged to be complete and was quenched by the
addition of saturated aqueous NaCl (30 mL) and H$_2$O (20 mL). The reaction mixture was poured
into a separatory funnel and the product was extracted with Et$_2$O (3 x 50 mL). The combined
organics were washed with saturated aqueous NaCl (50 mL), dried over \ce{Na2SO4}, filtered, and
concentrated. Purification by flash column chromatography (85\% Et$_2$O in pentane v/v) afforded the
desired product as a white solid (3.34 g, 94.5\%). Enantioenrichment was achieved by
recrystallization from hot Et$_2$O and hexanes (approx. 3:1 v/v) to afford the optically pure
product \ref{cmp:xbah} (2.14 g, 60.6\%) with 99\% ee (AS-H, 50 \degc, 150 psi, 3.0 mL/min, 3\% MeOH,
$\lambda$ = 220 nm; t$_R$ = 16.27 min (major), 18.03 min (minor)).\\
R$_f$ = 0.38 (60\% ethyl acetate in hexanes v/v); $^1$H NMR (CDCl$_3$, 500 MHz) $\delta$ 3.83 (ddd,
\textit{J} = 13.2, 7.3, 5.9 Hz, 1H), 2.62-2.52 (m, 2H), 2.46-2.36 (m, 2H), 2.19-2.11 (m, 1H), 2.07 (ddd, \textit{J} = 12.7, 5.4, 2.0 Hz, 1H), 1.88-1.74
(m, 2H), 1.66 (dd, \textit{J} = 1.2 Hz, 3H), 1.32 (s, 3H); $^{13}$C NMR (CDCl$_3$, 125 MHz) $\delta$
198.96, 168.10, 129.09, 81.05, 45.15, 34.11, 33.41, 29.60, 25.76, 15.34, 10.80; HRMS (ESI+) Calcd.
for \ce{C11H17O2} [M+H]$^+$: 181.1229; Found 181.1220.
\begin{figure}[h]
\vspace{10pt}
\centering
\includegraphics[width=5.25in]{chp_singlecarbon/images/sfc/xbah_sfc.jpg}
\caption{SFC trace for \CMPxbah~(\ref{cmp:xbah})}
\vspace{-10pt}
\end{figure}
%***************[xbah]%***************%

\pagebreak
%***************[xbai]%***************%
\begin{wrapfigure}{l}{1.0in}
  \vspace{-12pt}
  \begin{center}
    \includegraphics[scale=0.8]{chp_singlecarbon/images/xbai}
    \begin{textblock}{1}(2.1,-0.7) \cmp{xbai} \end{textblock}
  \end{center}
  \vspace{-30pt}
\end{wrapfigure}\noindent \textbf{\CMPxbai}\ (\ref{cmp:xbai}). Racemic Hajos-Parrish
ketone \ref{cmp:xbag}\footnote{Prepared in an identical fashion to \ref{cmp:xbag} with
DL-phenylalanine.} (100 mg, 0.563 mmol, 1.00 equiv) was dissolved in 2.0 mL of EtOH, and the
resulting homogeneous solution was cooled to $-$25 \degc. Sodium borodeuteride (7.4 mg, 0.18 mmol, 0.31 equiv) was added as a solid, and the
mixture was closely monitored by TLC. After 20 minutes, the reaction was judged to be
complete and was quenched by the addition of saturated aqueous NaCl (10 mL) and H$_2$O (10
mL). The reaction mixture was poured into a separatory funnel and the product was extracted
with Et$_2$O (3 x 15 mL). The combined organics were washed with saturated aqueous NaCl (25
mL), dried over \ce{Na2SO4}, filtered, and concentrated. Purification by flash column chromatography
(85\% Et$_2$O in pentane v/v) afforded the desired product as a white solid (113 mg, quantitative),
mp 67-73 \degc \\
R$_f$ = 0.36 (85\% Et$_2$O in pentane v/v); $^1$H NMR (CDCl$_3$, 500 MHz) $\delta$ 2.61-2.52 (m,
2H), 2.45-2.35 (m, 2H), 2.17-2.11 (m, 1H), 2.07 (ddd, \textit{J} = 12.9, 5.4, 1.6 Hz, 1H), 1.86-1.74 (m, 2H), 1.66 (s, 3H), 1.60 (s, 1H), 1.11 (s, 3H); $^{13}$C NMR (CDCl$_3$, 125 MHz) $\delta$ 199.07, 168.37, 128.98, 80.48 (t, J =
20.7 Hz), 45.03, 34.01, 33.39, 29.40, 25.76, 15.31, 10.76; IR (neat) 3415 (bm), 2922 (bm), 1641
(s), 1354 (m), 1326 (m), 1298 (w), 1171 (m), 1126 (m), 1044 (bm) cm$^{-1}$; HRMS (ESI+) Calcd. for \ce{C11H16DO2} [M+H]$^+$: 182.1291; Found 182.1298.
%***************[xbai]%***************%

\vspace{10pt}
%%%%%%%%%%%%%%%%%%%%%%%%%%%%%%%
% Reductive alkylation
%%%%%%%%%%%%%%%%%%%%%%%%%%%%%%%
\noindent\textbf{General procedure for dissolved metal reductive alkylation:}
A flame-dried, 2-neck, 25 mL round bottom
flask equipped with a cold finger condenser, septum, and a magnetic stir bar was charged with
lithium wire (5.8 mg, 0.83 mmol, 3.0 equiv), and the entire apparatus was flame-dried again.
After backfilling with argon, the apparatus was cooled to $-$78 \degc, and ammonia (3.6 mL) was
freshly distilled from sodium metal into the reaction flask, dissolving the lithium wire and
forming a deep blue solution. A solution of enone \ref{cmp:xbah} (50.0 mg, 0.277 mmol, 1.00 equiv)
in 2.0 mL of THF was then added to the dissolved metal solution over 30 minutes \textit{via} syringe
pump.
Upon completion of the addition, the reaction mixture was warmed to $-$25 \degc\  and stirred at
this temperature for 1 hour. The solution was then re-cooled to $-$78 \degc\  and diluted with 1.2
mL of THF. In a separate flask, a solution of the appropriate electrophile (1.39 mmol, 5.0 equiv) in
1.6 mL THF was pre-cooled to $-$78 \degc\  and then added as rapidly as possible to the blue
solution \textit{via} syringe.
Almost immediately, the deep blue color bleached to white, and stirring was continued at $-$
78 \degc\  for 8 hours. The reaction mixture was then warmed slowly to room temperature, and the
ammonia was allowed to evaporate from the reaction mixture. During this time, pressure generated from the vaporization of ammonia was liberated through an exit needle or through an
external bubbler. The basic solution was acidified by the addition of 20 mL of saturated aqueous
NH$_4$Cl. The mixture was poured into a separatory funnel, and the product was extracted with
Et$_2$O (3 x 20 mL). The combined organics were washed with H$_2$O (15 mL), saturated aqueous
NaCl (15 mL), dried over \ce{Na2SO4}, filtered, and concentrated to afford the crude product.
Purification was carried out by flash column chromatography on silica gel (ethyl acetate in
hexanes). \textit{Note:} This reaction must be carried out under an atmosphere of argon gas, as the
use of nitrogen gas results in reaction with lithium metal to form considerable amounts of lithium nitride (Li$_3$N).

\vspace{10pt}
%***************[xban]%***************%
\begin{wrapfigure}{l}{1.3in}
  \vspace{-28pt}
  \begin{center}
    \includegraphics[scale=0.8]{chp_singlecarbon/images/xban}
  \end{center}
  \vspace{-30pt}
\end{wrapfigure}\noindent \textbf{\CMPxban}\ (\ref{cmp:xban}). Carried out according to the general procedure
for dissolved metal reductive alkylation with enone \ref{cmp:xbah} (56.2 mg, 0.312 mmol,
1.00 equiv) and iodide \ref{cmp:xbam} (488 mg, 1.56 mmol, 5.00 equiv). The
electrophile was not pre-cooled, however, due to a lack of solubility
below room temperature. Purification by flash column chromatography
(50\% ethyl acetate in hexanes v/v) afforded the desired product \ref{cmp:xban} as a white solid
(92.6 mg, 80.9\%), mp 165-168 \degc. \\
\rotation = $-$27.64 (c 1.13, CHCl$_3$); R$_f$ = 0.33 (60\% ethyl acetate in hexanes); $^1$H NMR (CDCl$_3$,
500 MHz) $\delta$ 6.37 (d, \textit{J} = 2.7 Hz, 1H), 6.15 (d, \textit{J} = 2.7 Hz, 1H), 3.86-3.79 (m, 4H), 3.74 (s, 3H),
3.51 (d, \textit{J} = 13.9 Hz, 1H), 3.0 (d, \textit{J} = 13.9 Hz, 1H), 2.87 (ddd, \textit{J} = 15.2, 9.3, 5.6 Hz, 1H), 2.38-
2.30 (m, 1H), 2.22 (ddd, \textit{J} = 11.0, 8.3, 0 Hz, 1H), 2.07-1.99 (m, 1H), 1.97-1.76 (m, 3H), 1.34 (s, 3H), 1.17 (dddd, \textit{J} = 9.3, 9.3, 9.3, 9.3 Hz, 1H), 0.90 (s, 3H); $^{13}$C NMR (CDCl$_3$, 125 MHz) $\delta$ 215.37, 158.20, 155.88, 137.45, 115.56, 107.95, 98.29, 81.39, 56.29, 55.79, 55.55, 52.32,
42.94, 41.50, 35.43, 32.59, 31.43, 26.82, 23.71, 19.71; IR (neat) 3448 (bm), 2958 (bm), 2878 (m),
1697 (s), 1590 (s), 1455 (s), 1330 (s), 1203 (s), 1163 (s), 1084 (m), 979 (m), 753 (m) cm$^{-1}$;
HRMS (ESI+) Calcd. for \ce{C20H28ClO4} [M+H]$^+$: 367.1676; Found 367.1684.
%***************[xban]%***************%

\vspace{10pt}
%***************[xbaz]%***************%
\begin{wrapfigure}{l}{1.15in}
  \vspace{-25pt}
  \begin{center}
    \includegraphics[scale=0.8]{chp_singlecarbon/images/xbaz}
  \end{center}
  \vspace{-30pt}
\end{wrapfigure}\noindent \textbf{\CMPxbaz}\ (\ref{cmp:xbaz}). Carried out according to the general procedure
for dissolved metal reductive alkylation with enone \ref{cmp:xbah} (499 mg, 2.77 mmol, 1.00
equiv) and iodide \ref{cmp:xbay} (5.40 mg, 13.9 mmol, 5.00 equiv). Purification by
flash column chromatography (40 to 70\% ethyl acetate in hexanes v/v)
afforded the desired product \ref{cmp:xbaz} as a white solid (968 mg, 78.9\%), mp 45-50 \degc.\\
\rotation = $-$32.50 (c 0.86, CHCl$_3$); R$_f$ = 0.33 (50\% ethyl acetate in hexanes); $^1$H NMR (CDCl$_3$,
500 MHz) $\delta$ 7.40-7.31 (m, 5H), 7.05 (d, \textit{J} = 8.8 Hz, 1H), 6.76 (d, \textit{J} = 8.8 Hz, 1H), 5.04 (d, J =
11.2 Hz, 1H), 4.89 (d, \textit{J} = 11.5 Hz, 1H), 3.83 (s, 3H), 3.71 (ddd, \textit{J} = 6.1, 6.1, 6.1 Hz, 1H), 3.42 (d,
\textit{J} = 13.7 Hz, 1H), 2.87 (d, \textit{J} = 13.7 Hz, 1H), 2.65 (dddd, \textit{J} = 13.7 Hz, 5.4, 5.4, 5.4 Hz, 1H), 2.10
(dd, \textit{J} = 11.7, 8.1 Hz, 1H), 2.04-1.90 (m, 2H), 1.83-1.74 (m, 1H), 1.73-1.62 (m, 1H), 1.56-1.47
(m, 2H), 1.47-1.38 (m, 1H), 1.17 (s, 3H), 1.08-0.93 (m, 1H), 0.79 (s, 3H); $^{13}$C NMR (CDCl$_3$, 125
MHz) $\delta$ 214.79, 151.12, 147.91, 137.57, 130.44, 128.75, 128.57, 128.34, 127.24, 124.37, 111.91,
81.00, 75.17, 57.00, 56.07, 51.83, 42.27, 37.15, 35.00, 32.25, 31.45, 27.15, 23.91, 19.08; IR
(neat) 3430 (bw), 2956 (bm), 2872 (bm), 1697 (s), 1576 (w), 1463 (bs), 1375 (m), 1275 (s), 1214
(bm), 1077 (bm), 974 (bs), 798 (m), 749 (s), 697 (s) cm$^{-1}$; HRMS (ESI+) Calcd. for
\ce{C26H32ClO4} [M+H]$^+$: 443.1989; Found 443.2005.
%***************[xbaz]%***************%

%***************[xbbo]%***************%
\begin{wrapfigure}{l}{1.25in}
  \vspace{-15pt}
  \begin{center}
    \includegraphics[scale=0.8]{chp_singlecarbon/images/xbbo}
  \end{center}
  \vspace{-30pt}
\end{wrapfigure}\noindent \textbf{\CMPxbbo}\ (\ref{cmp:xbbo}). Carried out according to the general procedure
for dissolved metal reductive alkylation with enone \ref{cmp:xbai} (193
mg, 1.06 mmol, 1.00 equiv) and iodide \ref{cmp:xbbn} (2.08 g, 5.31 mmol, 5.00
equiv). Purification by flash column chromatography (30 to 60\% ethyl
acetate in hexanes v/v) afforded the desired product \ref{cmp:xbbo} as a white solid
(398 mg, 84.1\%), mp 44-51 \degc. \\
R$_f$ = 0.33 (50\% ethyl acetate in hexanes); $^1$H NMR (CDCl$_3$, 500 MHz) $\delta$ 7.40-7.30 (m, 5H), 7.04
(d, \textit{J} = 8.8 Hz, 1H), 6.76 (d, \textit{J} = 8.8 Hz, 1H), 5.03 (d, \textit{J} = 11.2 Hz, 1H), 4.89 (d, \textit{J} = 11.2 Hz, 1H),
3.83 (s, 3H), 2.63 (ddd, \textit{J} = 16.6, 6.4, 6.4 Hz, 1H), 2.09 (dd, \textit{J} = 11.7, 8.1 Hz, 1H), 2.03-1.90 (m,
2H), 1.81-1.74 (m, 1H), 1.72-1.47 (m, 3H), 1.45-1.38 (m, 1H), 1.16 (s, 3H), 1.03-0.92 (m, 1H),
0.79 (s, 3H);
13
C NMR (CDCl$_3$, 125 MHz) $\delta$ 214.80, 151.12, 147.92, 137.58, 130.38, 128.74,
128.56, 128.33, 127.22, 124.36, 111.94, 80.50 (t, \textit{J} = 21.9 Hz), 75.16, 56.98, 56.07, 51.68, 42.14,
35.00, 32.21, 31.36, 27.12, 23.89, 19.02; IR (neat) 3449 (bw), 3063 (bw), 2956 (bm), 2870 (bm),
1698 (m), 1574 (w), 1461 (bs), 1374 (m), 1293 (m), 1267 (m), 1097 (bm), 974 (bs), 798 (m), 749
(bm), 698 (s) cm$^{-1}$; HRMS (ESI+) Calcd. for \ce{C26H29D3ClO4} [M+H]$^+$: 446.2177; Found 446.2190.
%***************[xbbo]%***************%

%%%%%%% End of reductive alkylation products

\vspace{10pt}
%***************[xbaj]%***************%
\begin{wrapfigure}{l}{1.3in}
  \vspace{-25pt}
  \begin{center}
    \includegraphics[scale=0.8]{chp_singlecarbon/images/xbaj}
    \begin{textblock}{1}(0.2,-3) \cmp{xbaj} \end{textblock}
  \end{center}
  \vspace{-30pt}
\end{wrapfigure}\noindent \textbf{\CMPxbaj}\ (\ref{cmp:xbaj}). In a drybox, LiAlH$_4$ (2.08 g,
54.9 mmol, 1.00 equiv) was weighed into a 250 mL round bottom flask
equipped with a magnetic stirbar. After removing the flask from the drybox,
50 mL of THF was added, and the resulting grey suspension was cooled to
0 \degc. In a separate flask, 3,5-dimethoxybenzoic acid (10.0 g, 54.9 mmol. 1.00 equiv) was
suspended in 60 mL of THF. The slurry of 3,5-dimethoxybenzoic acid was added to the LiAlH$_4$
suspension via syringe, and the reaction mixture was allowed to warm slowly to room
temperature and stir for 12 hours. The dark grey solution was then re-cooled to 0 \degc, and H$_2$O
(5 mL) was slowly added to quench the excess LiAlH$_4$. The resulting thick slurry was warmed to
room temperature and diluted with 50 mL of 1 N HCl. The reaction mixture was poured into a
separatory funnel and the product was extracted with Et$_2$O (3 x 60 mL). The combined organics
were washed with H$_2$O (2 x 100 mL), saturated aqueous NaCl (100 mL), dried over \ce{Na2SO4},
filtered, and concentrated to give \ref{cmp:xbaj} as a white solid that was used without any further
purification (9.00 g, 97.5\%), mp 45-48 \degc.\\
R$_f$ = 0.16 (30\% ethyl acetate in hexanes v/v); $^1$H NMR (CDCl$_3$, 500 MHz) $\delta$ 6.53 (d, \textit{J} = 2.1 Hz, 2H),
6.39 (t, \textit{J} = 2.1 Hz, 1H), 4.64 (d, \textit{J} = 6.1 Hz, 2H), 3.80 (s, 6H), 1.66 (t, \textit{J} = 6.1 Hz, 1H); $^{13}$C NMR
(CDCl$_3$, 125 MHz) $\delta$ 160.92, 143.53, 104.57, 99.51, 65.11, 55.33; IR (neat) 3346 (bm), 2938
(bm), 2838 (m), 1594 (s), 1458 (s), 1428 (s), 1317 (m), 1202 (s), 1148 (s), 1058 (s), 1034 (s), 829
(s), 688 (m) cm$^{-1}$; HRMS (ESI+) Calcd. for \ce{C9H13O3} [M+H]$^+$: 169.0865; Found 169.0863.
%***************[xbaj]%***************%

\vspace{10pt}
%***************[xbak]%***************%
\begin{wrapfigure}{l}{1.3in}
  \vspace{-25pt}
  \begin{center}
    \includegraphics[scale=0.8]{chp_singlecarbon/images/xbak}
  \end{center}
  \vspace{-30pt}
\end{wrapfigure}\noindent \textbf{\CMPxbak}\ (\ref{cmp:xbak}). To a solution
of \ref{cmp:xbaj} (10.4 g, 62.0 mmol, 1.00 equiv) in 310 mL of CCl$_4$ was added \textit{N}-
chlorosuccinimide (7.86 g, 58.9 mmol, 0.950 equiv) as a solid. The solution
was then refluxed for 48 hours. The reaction mixture was cooled to room
temperature and concentrated to remove CCl$_4$. The resulting residue was suspended in 200 mL of
Et$_2$O and filtered through a sintered glass frit. The filtrate was then washed with saturated
aqueous NaHCO$_3$ (100 mL), saturated aqueous NH$_4$Cl (100 mL), H$_2$O (100 mL), and saturated
aqueous NaCl (100 mL). The extract was then dried over \ce{Na2SO4}, filtered, and concentrated. The
crude solid was recrystallized from hot Et$_2$O and hexanes (approx. 5:1 v/v) to afford the desired
product \ref{cmp:xbak} as a white crystalline solid (9.00 g, 71.7\%), mp 88-90 \degc.\\
R$_f$ = 0.24 (30\% ethyl acetate in hexanes v/v); $^1$H NMR (CDCl$_3$, 500 MHz) $\delta$ 6.68 (d, \textit{J} = 2.8 Hz, 1H),
6.47 (d, \textit{J} = 2.8 Hz, 1H), 4.77 (d, \textit{J} = 6.7 Hz, 2H), 3.88 (s, 3H), 3.83 (s, 3H), 1.95 (t, 6.7 Hz, 1H);
$^{13}$C NMR (CDCl$_3$, 125 MHz) $\delta$ 159.22, 155.76, 140.22, 112.26, 104.22, 99.04, 63.03,
56.33, 55.68; IR (neat) 3282 (bm), 2937 (m), 2838 (m), 1590 (s), 1454 (s), 1420 (s), 1330 (s), 1198 (s),
1084 (s), 1030 (s), 952 (m), 831 (s), 680 (m), 604 (s) cm$^{-1}$; HRMS (ESI+) Calcd. for
\ce{C9H12ClO3} [M+H]$^+$: 203.0475; Found 203.0470.
%***************[xbak]%***************%

\vspace{10pt}
%***************[xbal]%***************%
\begin{wrapfigure}{l}{1.3in}
  \vspace{-25pt}
  \begin{center}
    \includegraphics[scale=0.8]{chp_singlecarbon/images/xbal}
    \begin{textblock}{1}(0.2,-3) \cmp{xbal} \end{textblock}
  \end{center}
  \vspace{-30pt}
\end{wrapfigure}\noindent \textbf{\CMPxbal}\ (\ref{cmp:xbal}). Benzyl
alcohol \ref{cmp:xbak} (2.80 g, 13.8 mmol, 1.00 equiv) was dissolved in 46 mL of
benzene, and the resulting homogeneous solution was cooled to 4 \degc.
PBr$_3$ (0.49 mL, 5.1 mmol, 0.37 equiv) was added dropwise,  then the
reaction mixture was warmed to room temperature and stirred for 2.5 hours. The reaction was
quenched by the addition of 50 mL of H$_2$O and transferred into a separatory funnel.
The product was extracted with Et$_2$O (3 x 50 mL) and the combined organics were washed with
H$_2$O (50 mL), saturated aqueous NaCl (50 mL), dried over \ce{Na2SO4}, filtered, and concentrated
to afford the desired product \ref{cmp:xbal} as a white solid that was used without further
purification (3.01 g, 82.1\%), mp 100-102 \degc.\\
R$_f$ = 0.37 (15\% ethyl acetate in hexanes v/v); $^1$H NMR (CDCl$_3$, 500 MHz) $\delta$ 6.58 (d, \textit{J} = 2.8 Hz, 1H),
6.48 (d, \textit{J} = 2.8 Hz, 1H), 4.57 (s, 2H), 3.88 (s, 3H), 3.82 (s, 3H); $^{13}$C NMR (CDCl$_3$, 125
MHz) $\delta$ 158.94, 156.25, 136.90, 114.59, 106.69, 100.26, 56.41, 55.71, 31.06; IR (neat) 3097 (w),
2975 (m), 1585 (s), 1470 (s), 1432 (s), 1334 (s), 1200 (s), 1165 (s), 1082 (s), 1030 (s), 951 (s),
819 (s), 721 (m), 673 (s) 610 (m) cm$^{-1}$; HRMS (ESI+) Calcd. for
\ce{C9H11}$^{79}$Br$^{37}$\ce{ClO2} [M+H]$^+$: 266.9601; Found 266.9601.
%***************[xbal]%***************%

\vspace{10pt}
%***************[xbam]%***************%
\begin{wrapfigure}{l}{1.3in}
  \vspace{-25pt}
  \begin{center}
    \includegraphics[scale=0.8]{chp_singlecarbon/images/xbam}
  \end{center}
  \vspace{-30pt}
\end{wrapfigure}\noindent \textbf{\CMPxbam}\ (\ref{cmp:xbam}). To a solution
of benzyl bromide \ref{cmp:xbal} (502 mg, 1.89 mmol, 1.00 equiv) in 3.2 mL of
acetone at room temperature, NaI (566 mg, 3.78 mmol, 2.00 equiv) was
added as a solid. The resulting suspension was stirred for 12 hours in the
dark. The reaction mixture was poured into 50\% aqueous \ce{Na2S2O3} (15 mL), and the product was
extracted with Et$_2$O (3 x 15 mL). The combined organics were washed with saturated aqueous
NaCl (15 mL), dried over \ce{Na2SO4}, filtered, and concentrated to afford the desired product
\ref{cmp:xbam} as a white solid that was used without further purification (539 mg, 91.3\%), mp
127-129 \degc.\\
R$_f$ = 0.37 (15\% ethyl acetate in hexanes v/v); $^1$H NMR (CDCl$_3$, 500 MHz) $\delta$ 6.54 (d, \textit{J} = 2.7 Hz, 1H),
6.44 (d, \textit{J} = 2.7 Hz, 1H), 4.50 (s, 2H), 3.87 (s, 3H), 3.80 (s, 3H);
$^{13}$C NMR (CDCl$_3$, 125 MHz) $\delta$ 158.87, 156.39, 138.31, 114.16, 105.99, 99.86, 56.41,
55.72, 3.05; IR (neat) 3058 (w), 2939 (w), 1586 (s), 1469 (s), 1418 (m), 1332 (s), 1204 (s), 1156 (s), 1076 (s), 1030 (m), 951 (m),
818 (m), 675 (m) cm$^{-1}$; HRMS (ESI+) Calcd. for \ce{C9H11ClIO2} [M+H]$^+$: 312.9492; Found
312.9490.
%***************[xbam]%***************%

\vspace{10pt}
%***************[xbao]%***************%
\begin{wrapfigure}{l}{1.35in}
  \vspace{-25pt}
  \begin{center}
    \includegraphics[scale=0.8]{chp_singlecarbon/images/xbao}
  \end{center}
  \vspace{-30pt}
\end{wrapfigure}\noindent \textbf{\CMPxbao}\ (\ref{cmp:xbao}). In a drybox, NaH (35.1 mg, 1.46 mmol,
7.02 equiv) was weighed into a 2-neck, 25 mL round bottom flask equipped with a magnetic stirbar.
After removing the flask from the drybox, a reflux condenser was installed. 1.5 mL of DMSO was
added, and the suspension was heated to 75 \degc\  for 1 hour. During this time, the reaction became
homogeneous, forming a teal-colored, clear solution. This solution was cooled to room temperature,
and a solution of \ce{Ph3PCH3I} (764 mg, 1.88 mmol, 9.04 equiv) in 2.6 mL of DMSO was added over 30
minutes via syringe pump. Upon addition of the salt, the reaction mixture became bright yellow.
After completion of the addition, the mixture was stirred for an additional 30 minutes at room
temperature, at which point a solution of racemic keto-alcohol \ref{cmp:xban} (76.3 mg, 0.208 mmol,
1.00 equiv) in 0.58 mL of DMSO was added dropwise. The reaction mixture was then heated to 75 \degc\ 
and stirred for 16 hours. The resulting amber solution was cooled to room temperature and acidified by the addition
of 5 mL of saturated aqueous NH$_4$Cl. The reaction mixture was diluted with H$_2$O (15 mL), poured
into a separatory funnel, and the product was extracted with Et$_2$O (3 x 15 mL). The combined
organics were washed with H$_2$O (15 mL), saturated aqueous NaCl (15 mL), dried over \ce{Na2SO4},
filtered, and concentrated. Purification by flash column chromatography (40\% Et$_2$O in pentane
v/v) afforded compound \ref{cmp:xbao} as a white solid (61.3 mg, 80.8\%), mp 117-119 \degc.\\
R$_f$ = 0.33 (60\% Et$_2$O in pentane v/v); $^1$H NMR (CDCl$_3$, 500 MHz) $\delta$ 6.69 (d, \textit{J} = 2.7 Hz, 1H), 6.38
(d, \textit{J} = 2.7 Hz, 1H), 4.75-4.71 (m, 2H), 3.86 (s, 3H), 3.78 (s, 3H), 3.61-3.57 (m, 1H), 2.97-2.89
(m, 2H), 2.57-2.47 (m, 1H), 2.42-2.32 (m, 1H), 2.09-1.90 (m, 2H), 1.82-1.65 (m, 4H), 1.57-1.50
(m, 1H), 1.49 (d, \textit{J} = 4.6 Hz, 1H), 1.11 (s, 3H), 0.67 (s, 3H); $^{13}$C NMR (CDCl$_3$, 125
MHz) $\delta$ 159.86, 158.13, 155.71, 139.15, 115.83, 108.95, 101.44, 97.62, 72.48, 56.27, 55.57,
51.66, 45.02, 41.99, 38.36, 31.34, 30.57, 27.79, 25.55, 23.26, 18.40; IR (neat) 3556 (bw), 2949
(bm), 1588 (s), 1454 (s), 1287 (m), 1201 (s), 1161 (s), 1082 (m), 1034 (s), 907 (m), 730 (s), 632
(w) cm$^{-1}$; HRMS (ESI+) Calcd. for \ce{C21H28ClO2} [M-OH]$^+$: 347.1778; Found 347.1766.
%***************[xbao]%***************%

\vspace{10pt}
%***************[xbap]%***************%
\begin{wrapfigure}{l}{1.45in}
  \vspace{-25pt}
  \begin{center}
    \includegraphics[scale=0.8]{chp_singlecarbon/images/xbap}
    \begin{textblock}{1}(3.2,-1) \cmp{xbap} \end{textblock}
  \end{center}
  \vspace{-30pt}
\end{wrapfigure}\noindent \textbf{\CMPxbap}\ (\ref{cmp:xbap}). To a solution of keto-
alcohol \ref{cmp:xbao} (170 mg, 0.464 mmol, 1.00 equiv) in 11.6 mL of
\ce{CH2Cl2}, \ce{Et3N} (129 $\mu$L, 0.928 mmol, 2.00 equiv) was added. The
solution was cooled to $-$78 \degc\  and treated dropwise with TBSOTf
(160 $\mu$L, 0.696 mmol, 1.50 equiv) \textit{via} syringe. The solution was stirred
for 2.5 hours at $-$78 \degc, after which the reaction was quenched by the addition of saturated
aqueous NH$_4$Cl (5 mL). After warming to room temperature, the mixture was poured into a
separatory funnel and the product was extracted with \ce{CH2Cl2} (3 x 10 mL). The combined
organics were washed with H$_2$O (10 mL), saturated aqueous NaCl (10 mL), dried over \ce{Na2SO4},
filtered, and concentrated. Purification by flash column chromatography (15\% ethyl acetate in
hexanes v/v) afforded the desired silyl ether \ref{cmp:xbap} as a white solid (178 mg, 79.9\%), mp
136-138 \degc.\\
\rotation = $-$30.54 (c 0.96, CHCl$_3$); R$_f$ = 0.32 (15\% ethyl acetate in hexanes v/v); $^1$H NMR (CDCl$_3$,
500 MHz) $\delta$ 6.37 (d, \textit{J} = 2.9 Hz, 1H), 6.18 (d, \textit{J} = 2.7 Hz, 1H), 3.84 (s, 3H), 3.76-3.72 (m, 4H),
3.50 (d, \textit{J} = 14.2 Hz, 1H), 2.98 (d, \textit{J} = 14.2 Hz, 1H), 2.78 (ddd, \textit{J} = 17.1, 8.3, 5.6 Hz, 1H), 2.33
(ddd, \textit{J} = 16.7, 8.1, 5.6 Hz, 1H), 2.15 (ddd, \textit{J} = 11.3, 8.3, 0 Hz, 1H), 1.98-1.73 (m, 4H), 1.53-1.43
(m, 1H), 1.23 (s, 3H), 1.15-1.05 (m, 1H), 0.92-0.90 (m, 12H), 0.05 (s, 3H), 0.05 (s, 3H); $^{13}$C
NMR (CDCl$_3$, 125 MHz) $\delta$ 215.75, 158.18, 155.86, 137.64, 115.61, 107.99, 98.30, 81.05, 56.29,
55.55, 55.36, 52.46, 43.11, 41.36, 35.66, 32.50, 31.96, 26.70, 26.00, 24.91, 19.76, 18.24, $-$4.19, $-$4.76; IR (neat) 2995 (s), 2857 (m), 1704 (s), 1591 (s), 1459 (s), 1332 (m), 1164 (s), 1081 (m), 835
(s), 775 (m) cm$^{-1}$; HRMS (ESI+) Calcd. for \ce{C26H42ClO4Si} [M+H]$^+$: 481.2541; Found 481.2530.
%***************[xbap]%***************%

\vspace{10pt}
%***************[xbaq]%***************%
\begin{wrapfigure}{l}{1.45in}
  \vspace{-25pt}
  \begin{center}
    \includegraphics[scale=0.8]{chp_singlecarbon/images/xbaq}
  \end{center}
  \vspace{-30pt}
\end{wrapfigure}\noindent \textbf{\CMPxbaq}\ (\ref{cmp:xbaq}). In a drybox, NaH
(32.6 mg, 1.36 mmol, 7.39 equiv) was weighed into a 2-neck, 25 mL
round bottom flask equipped with a magnetic stirbar. After removing
the flask from the drybox, a reflux condenser was installed. DMSO (1.5 mL) was added, and the
suspension was heated to 75 \degc\  for 1 hour.
During this time, the reaction became homogeneous, forming a teal-colored, clear solution. This
solution was cooled to room temperature, and a solution of \ce{Ph3PCH3Br} (624 mg, 1.75 mmol, 9.51
equiv) in 2.4 mL of DMSO was added over 30 minutes \textit{via} syringe pump. Upon addition of the
salt solution, the reaction mixture became bright yellow. After completion of the addition, the
mixture was stirred for an additional 30 minutes at room temperature, at which point a solution of
ketone \ref{cmp:xbap} (93.4 mg, 0.184 mmol, 1.00 equiv) in 0.56 mL of DMSO and 0.50 mL of THF
was added dropwise. The reaction mixture was then heated to 75 \degc\  and stirred for 16 hours. The
resulting amber solution was cooled to room temperature and acidified by the addition of 5 mL of
saturated aqueous NH$_4$Cl. The reaction mixture was diluted with H$_2$O (15 mL), poured into a
separatory funnel, and the product was extracted with Et$_2$O (3 x 15 mL). The combined organics
were washed with H$_2$O (15 mL), saturated aqueous NaCl (15 mL), dried over \ce{Na2SO4}, filtered,
and concentrated. Purification by flash column chromatography (5\% Et$_2$O in pentane v/v) afforded
the desired olefin \ref{cmp:xbaq} as a white solid (95.0 mg, quantitative), mp 97-98 \degc.\\
\rotation = $+$32.36 (c 1.00, CHCl$_3$); R$_f$ = 0.33 (50\% Et$_2$O in pentane v/v); $^1$H NMR (CDCl$_3$, 500
MHz) $\delta$ 6.35 (d, \textit{J} = 2.7 Hz, 1H), 6.21 (d, \textit{J} = 2.7 Hz, 1H), 4.93 (s, 1H), 4.42 (s, 1H), 3.85 (s, 3H),
3.73 (s, 3H), 3.55 (dd, \textit{J} = 5.9, 1.5 Hz, 1H), 3.25 (d, \textit{J} = 13.5 Hz, 1H), 3.01 (d, \textit{J} = 13.4 Hz, 1H),
2.71 (ddd, \textit{J} = 13.7, 13.7, 5.4 Hz, 1H), 2.24-2.18 (m, 1H), 2.09-2.02 (m, 1H), 1.98-1.89 (m, 1H),
1.84-1.74 (m, 1H), 1.47-1.22 (m, 7H), 0.92 (s, 9H), 0.81 (s, 3H), 0.5 (s, 3H), 0.4 (s, 3H); $^{13}$C
NMR (CDCl$_3$, 125 MHz) $\delta$ 157.51, 155.43, 151.32, 139.52, 115.84, 111.96, 108.01, 97.80, 84.33, 56.26, 55.42, 55.00, 45.96, 44.34, 41.51, 33.80, 31.57, 29.91, 26.68, 26.10, 23.31, 22.42, 18.34, $-$4.27, $-$4.69; IR (neat) 2953 (s), 2930 (s), 2856 (m), 1590 (s), 1455 (s), 1371 (m), 1285 (m), 1255
(m), 1202 (m), 1163 (s), 1074 (s), 1004 (m), 833 (s), 722 (m) cm$^{-1}$; HRMS (ESI+) Calcd. for
\ce{C27H44ClO3Si} [M+H]$^+$: 479.2748; Found 479.2733.
%***************[xbaq]%***************%

\vspace{10pt}
%***************[xbar]%***************%
\begin{wrapfigure}{l}{1.4in}
  \vspace{-25pt}
  \begin{center}
    \includegraphics[scale=0.8]{chp_singlecarbon/images/xbar}
    \begin{textblock}{1}(3,-1) \cmp{xbar} \end{textblock}
  \end{center}
  \vspace{-30pt}
\end{wrapfigure}\noindent \textbf{\CMPxbar}\ (\ref{cmp:xbar}). Exocyclic alkene \ref{cmp:xbaq} (1.00
g, 2.09 mmol, 1.00 equiv) and \ce{RhCl3.H2O} (87.3 mg, 0.417 mmol, 0.200 equiv) were
weighed into a 100 mL round bottom flask equipped with a magnetic
stirbar and dissolved in 21 mL of CHCl$_3$ and 21 mL of EtOH. The
resulting deep red solution was refluxed for a period of 2.5 days, during
which time the solution got darker and a metallic precipitate formed. The reaction mixture was
concentrated, and the crude residue was purified by flash column chromatography (60\% Et$_2$O in
pentane v/v) to afford the desired product \ref{cmp:xbar} as a white solid (749 mg, 98.2\%), mp
44-48 \degc. \\
\rotation = $-$130.65 (c 0.39, CHCl$_3$); R$_f$ = 0.36 (50\% Et$_2$O in pentane v/v); $^1$H NMR (CDCl$_3$, 500
MHz) $\delta$ 6.45 (d, \textit{J} = 2.7 Hz, 1H), 6.38 (d, \textit{J} = 2.7 Hz, 1H), 5.58-5.62 (m,
1H), 3.86 (s, 3H), 3.76 (s, 3H), 3.56 (ddd, \textit{J} = 8.8, 6.1, 6.1 Hz, 1H), 3.19 (d, \textit{J} = 12.9 Hz, 1H), 2.75 (d, \textit{J} = 13.2 Hz, 1H),
2.12-1.95 (m, 3H), 1.85 (dddd, \textit{J} = 6.1, 6.1, 6.1, 6.1 Hz, 1H), 1.73-1.66 (m, 1H), 1.46 (s, 3H),
1.42-1.32 (m, 2H), 1.16 (s, 3H), 1.09-0.99 (m, 1H), 0.94 (s, 3H); $^{13}$C NMR (CDCl$_3$, 125
MHz) $\delta$ 158.02, 155.65, 141.40, 139.51, 122.22, 116.08, 108.25, 97.96, 82.56, 56.29, 55.95,
55.53, 43.45, 42.90, 39.97, 34.98, 31.60, 26.58, 25.23, 22.14, 21.50; IR (neat) 3407 (bm), 2956
(m), 2873 (m), 1590 (s), 1455 (s), 1329 (m), 1202 (m), 1162 (s), 1118 (m), 1036 (m), 811 (w)
cm$^{-1}$; HRMS (ESI+) Calcd. for \ce{C21H30ClO3} [M+H]$^+$: 365.1884; Found 365.1879.
%***************[xbar]%***************%

%***************[xbas]%***************%
\begin{wrapfigure}{l}{1.4in}
  \vspace{-25pt}
  \begin{center}
    \includegraphics[scale=0.8]{chp_singlecarbon/images/xbas}
  \end{center}
  \vspace{-30pt}
\end{wrapfigure}\noindent \textbf{\CMPxbas}\ (\ref{cmp:xbas}). Alcohol \ref{cmp:xbar} (610 mg, 1.67
mmol, 1.00 equiv) and Celite$\textsuperscript{\textregistered}$ 545 (720 mg) were weighed into a 25
mL round bottom flask equipped with a magnetic stirbar and suspended in 8.4
mL of \ce{CH2Cl2}. PCC (721 mg, 3.34 mmol, 2.00 equiv) was then added as a
solid, causing a black discoloration, and the mixture was stirred at room
temperature for 2 hours. The reaction mixture was diluted with 50 mL of Et$_2$O, filtered through
Celite$\textsuperscript{\textregistered}$ 545 on a sintered glass frit, and concentrated. The crude
residue was purified by flash column chromatography (25\% Et$_2$O in pentanes v/v) to afford the
desired ketone \ref{cmp:xbas} as a white solid (561 mg, 92.6\%), mp 130-135 \degc. \\
\rotation = $-$99.36 (c 1.68, CHCl$_3$); R$_f$ = 0.29 (20\% \ce{Et2O} in pentane v/v); $^1$H NMR (CDCl$_3$, 500
MHz) $\delta$ 6.41 (d, \textit{J} = 2.7 Hz, 1H), 6.40 (d, \textit{J} = 2.7 Hz, 1H), 5.51-5.47 (m, 1H), 3.88 (s, 3H), 3.76
(s, 3H), 3.24 (d, \textit{J} = 13.2 Hz, 1H), 2.80 (d, \textit{J} = 13.2 Hz, 1H), 2.35-2.25 (m, 2H), 2.25-2.15 (m,
2H), 1.99-1.91 (m, 2H), 1.54-1.52 (m, 3H), 1.41-1.32 (m, 4H), 1.03 (s, 3H); $^{13}$C NMR (CDCl$_3$,
125 MHz) $\delta$ 223.25, 158.10, 155.82, 139.81, 138.97, 120.87, 116.05, 108.61, 97.88, 56.30, 55.54,
53.45, 47.24, 42.34, 40.94, 36.15, 31.39, 26.91, 23.98, 21.51, 21.27; IR (neat) 2964 (m), 2937
(m), 2839 (w), 1737 (s), 1590 (s), 1455 (s), 1330 (m), 1205 (m), 1163 (s), 1086 (m), 1036 (m)
cm$^{-1}$ ; HRMS (ESI+) Calcd. for \ce{C21H28ClO3} [M+H]$^+$: 363.1727; Found 363.1726.
%***************[xbas]%***************%

%***************[xbat]%***************%
\begin{wrapfigure}{l}{1.4in}
  \vspace{-25pt}
  \begin{center}
    \includegraphics[scale=0.8]{chp_singlecarbon/images/xbat}
  \end{center}
  \vspace{-30pt}
\end{wrapfigure}\noindent \textbf{\CMPxbat}\ (\ref{cmp:xbat}). To a solution of racemic
trisubstituted ene-one \ref{cmp:xbas} (18.6 mg, 0.0513 mmol, 1.00 equiv) in 0.3 mL of \ce{CH2Cl2} at
room temperature, PtO$_2$ (1.2 mg, 0.00513 mmol, 0.10 equiv) was added as a solid. With vigorous
stirring, the suspension was purged for 1 minute with hydrogen from a balloon, during which time the
brown PtO$_2$ turned black, signifying reduction to the active Pt(0). The reaction was stirred for
3.5 hours under a positive pressure of hydrogen and then filtered through
Celite$\textsuperscript{\textregistered}$ 545. After removal of solvent, $^1$H NMR analysis of the
crude mixture showed incomplete conversion, so the material was resubjected to the reaction conditions. After another 6 hours of stirring under hydrogen atmosphere, the
suspension was again filtered and concentrated. Purification by flash column chromatography
(15\% Et$_2$O, 75\% pentane, 10\% \ce{CH2Cl2} v/v/v) afforded the desired $\beta$-methyl product
\ref{cmp:xbat} as a white solid (12.1 mg, 64.6\%), mp 131-133 \degc. X-ray quality single crystals  were obtained by
crystallization from hot Et$_2$O and hexanes (approx. 5:1 v/v). \\
R$_f$ = 0.43 (30\% \ce{Et2O} in pentane v/v); $^1$H NMR (CDCl$_3$, 500 MHz) $\delta$ 6.42 (d, \textit{J} = 2.9 Hz, 1H), 6.41
(d, \textit{J} = 2.7 Hz, 1H), 3.89 (s, 3H), 3.80 (s, 3H), 3.02 (d, \textit{J} = 13.7 Hz, 1H), 2.63 (d, \textit{J} = 13.7 Hz,
1H), 2.32-2.23 (m, 1H), 2.09-1.97 (m, 3H), 1.88 (d, \textit{J} = 7.6 Hz, 1H) 1.79-1.70 (m, 1H), 1.53-1.45
(m, 1H), 1.33-1.27 (m, 1H), 1.20-1.04 (m, 2H), 1.00 (d, \textit{J} = 6.8 Hz, 3H), 0.93 (s, 3H), 0.71 (s,
3H); $^{13}$C NMR (CDCl$_3$, 125 MHz) $\delta$ 222.16, 158.01, 155.88, 138.93, 116.17, 108.63, 97.80,
56.33, 55.55, 49.60, 49.39, 42.14, 41.71, 37.07, 34.74, 30.23, 27.81, 27.41, 21.05, 17.37, 15.89;
IR (neat) 2961 (m), 2926 (m), 1732 (s), 1589 (s), 1454 (s), 1329 (m), 1202 (s), 1164 (s), 1087
(m), 1038 (m) cm$^{-1}$; HRMS (ESI+) Calcd. for \ce{C21H30ClO3} [M+H]$^+$: 365.1884; Found 365.1895.
%***************[xbat]%***************%

\vspace{10pt}
%***************[xbau]%***************%
\begin{wrapfigure}{l}{1.4in}
  \vspace{-25pt}
  \begin{center}
    \includegraphics[scale=0.8]{chp_singlecarbon/images/xbau}
  \end{center}
  \vspace{-30pt}
\end{wrapfigure}\noindent \textbf{\CMPxbau}\ (\ref{cmp:xbau}). Isolated from the hydrogenation
reaction above to afford the $\alpha$-methyl diastereomer  \ref{cmp:xbau} as a white solid (6.3 mg,
33.7\%), mp 147-149 \degc. X-ray quality single crystals  were obtained by
crystallization from hot Et$_2$O and hexanes (approx. 5:1 v/v). \\
R$_f$ = 0.31 (15\% \ce{Et2O}, 10\% \ce{CH2Cl2} in pentane v/v/v); $^1$H NMR (CDCl$_3$, 500
MHz) $\delta$ 6.46 (d, \textit{J} = 2.9 Hz, 1H), 6.40 (d, \textit{J} = 2.9 Hz, 1H), 3.88 (s, 3H),
3.78 (s, 3H), 3.07 (d, \textit{J} = 13.4 Hz, 1H), 2.82 (d, \textit{J} = 13.7 Hz, 1H), 2.41
(ddd, \textit{J} = 19.3, 8.5, 2.4 Hz, 1H), 2.28 (ddd, \textit{J} = 10.5, 7.0, 0 Hz, 1H), 2.19-2.10 (m, 1H), 1.87-1.74
(m, 2H), 1.74-1.65 (m, 1H), 1.51 (m, 3H), 1.38 (s, 3H), 1.27-1.21 (m, 1H), 1.10 (d, \textit{J} = 7.0 Hz, 3H), 0.76 (s, 3H); $^{13}$C NMR (CDCl$_3$, 125 MHz) $\delta$ 222.87, 158.05, 156.01, 139.90, 116.15,
109.01, 97.26, 56.32, 55.58, 49.76, 48.71, 39.01, 37.49, 35.86, 35.01, 28.95, 26.32, 23.85, 22.18,
20.86, 16.19; IR (neat) 2963 (m), 2877 (m), 1732 (s), 1590 (s), 1455 (s), 1327(m), 1201 (s), 1162
(s), 1092 (m), 1035 (m), 732 (m) cm$^{-1}$; HRMS (ESI+) Calcd. for \ce{C21H30ClO3} [M+H]$^+$: 365.1884;
Found 365.1885.
%***************[xbau]%***************%

\vspace{10pt}
%***************[xbav]%***************%
\begin{wrapfigure}{l}{1.4in}
  \vspace{-25pt}
  \begin{center}
    \includegraphics[scale=0.8]{chp_singlecarbon/images/xbav}
  \end{center}
  \vspace{-30pt}
\end{wrapfigure}\noindent \textbf{\CMPxbav}\ (\ref{cmp:xbav}). In a drybox, Sc(OTf)$_3$ (5.2 mg,
0.011 mmol, 0.052 equiv) was weighed directly into a 1.5 mL vial equipped with a
magnetic stirbar. A solution of ketone \ref{cmp:xbas} (76.6 mg, 0.211 mmol,
1.00 equiv) in CDCl$_3$ (0.8 mL) was transferred directly to the solid
Sc(OTf)$_3$. The cloudy gray suspension was stirred for 15 minutes at which
point TMSD (215 $\mu$L, 0.422 mmol, 2.00 equiv, 1.96 M in hexanes) was introduced dropwise. The
entire reaction mixture (including any residual solids) was transferred \textit{via} glass pipette
to a J.
Young NMR tube, and the vial was rinsed with an additional 0.2 mL of CDCl$_3$. The reaction tube
was removed from the drybox, connected to a nitrogen manifold, and placed in an oil bath pre-
heated to 50 \degc. After 16 hours of heating, the reaction was cooled to room temperature. $^1$H
NMR analysis indicated complete conversion and an approximate 8.5:1 ratio of regioisomeric silyl
products. The reaction mixture was rinsed from the NMR tube with Et$_2$O (5 mL) and
concentrated to give a crude yellow oil. The crude mixture was immediately dissolved in 4 mL of
1:1 (v/v) 1N HCl: THF and stirred for 2 hours. The reaction mixture was poured into saturated
NaHCO$_3$ (20 mL) and the products were extracted with Et$_2$O (3 x 20 mL). The combined organics
were washed with saturated aqueous NaCl (50 mL), dried over \ce{Na2SO4}, filtered, and
concentrated. Purification by flash column chromatography (18\% ethyl acetate in hexanes v/v)
provided the desired homologated ketone \ref{cmp:xbav} as a colorless oil (71.1 mg, 88.9\%) as well
 as the minor regioisomer as a colorless oil (8.6 mg, 10.8\%). \\
 \rotation = $+$4.12 (c 1.33, CHCl$_3$); R$_f$ = 0.35 (18\% ethyl acetate in hexanes); $^1$H NMR (CDCl$_3$, 500
MHz) $\delta$ 6.46 (d, \textit{J} = 2.7 Hz, 1H), 6.39 (d, \textit{J} = 2.7 Hz, 1H), 5.54-5.51 (m, 1H), 3.87 (s, 3H), 3.74
(m, 3H), 3.2–15 (d, \textit{J} = 14.4 Hz, 1H), 2.83 (d, \textit{J} = 14.4 Hz, 1H), 2.63-2.56 (m , 1H), 2.48-2.44 (m,
1H), 2.12 (ddd, \textit{J} = 8.1, 4.9, 0 Hz, 1H), 2.09-2.03 (m, 1H), 1.93-1.86 (m, 1H), 1.81-1.72 (m, 4H),
1.64-1.53 (m, 2H), 1.34 (s, 3H), 1.01 (s, 3H); $^{13}$C NMR (CDCl$_3$, 125 MHz) $\delta$ 216.38, 158.10,
155.76, 139.31, 138.47, 121.64, 115.92, 107.47, 97.82, 56.29, 55.52, 49.03, 48.89, 42.66, 40.15,
36.66, 32.58, 26.44, 24.27, 24.05, 23.22, 20.19; IR (neat) 2963 (m), 2940 (m), 1701 (s), 1590 (s),
1454 (s), 1330 (m), 1203 (s), 1163 (s), 1087 (m), 1036 (w), 830 (w) cm$^{-1}$; HRMS (ESI+) Calcd.
for \ce{C22H30ClO3} [M+H]$^+$: 377.1884; Found 377.1891.
 %***************[xbav]%***************%

\vspace{10pt}
%***************[xbaw]%***************%
\begin{wrapfigure}{l}{1.1in}
  \vspace{-25pt}
  \begin{center}
    \includegraphics[scale=0.8]{chp_singlecarbon/images/xbaw}
  \end{center}
  \vspace{-30pt}
\end{wrapfigure}\noindent \textbf{\CMPxbaw}\ (\ref{cmp:xbaw}). Benzyl protected reduced
\textit{o}-vanillin\footnote{Prepared in two steps from \textit{o}-vanillin according to the
literature procedure:
{\frenchspacing Speicher, A.; Holz, J. \textit{Tetrahedron Lett.} \textbf{2010},	\textit{51},
2986-2989.} \label{ref:holz}} (35.5 g, 145 mmol, 1.00 equiv) was weighed into a 500 mL round bottom
flask equipped with a magnetic stirbar and dissolved in 290 mL of \ce{CH2Cl2}. The solution was cooled to 0 \degc, and 1,3-dichloro-5,5-
dimethylhydantoin (34.4 g, 174 mmol, 1.20 equiv) was added as a solid. The reaction mixture was then
stirred for 12 hours at 4 \degc. The resulting slurry was diluted with saturated aqueous
\ce{Na2S2O3} (150 mL), and the product was extracted with \ce{CH2Cl2} (3 x 100 mL). The combined
organics were washed with saturated aqueous NaHCO$_3$ (300 mL), H$_2$O (300 mL), saturated aqueous
NaCl (300 mL), dried over MgSO$_4$, filtered, and concentrated. The crude solid was recrystallized
from hot hexanes and ethyl acetate (approx. 10:1 v/v) to afford the desired product \ref{cmp:xbaw}
as a white crystalline solid (34.5 g, 85.8\%), mp 80-83 \degc. The mother liquor was then purified
by column chromatography (40\% ethyl acetate in hexanes v/v) to provide more of the desired compound
as a white solid (2.51 g, 6.2\%).\\
R$_f$ = 0.30 (40\% ethyl acetate in hexanes v/v); $^1$H NMR (CDCl$_3$, 500 MHz) $\delta$ 7.46-7.43 (m, 2H), 7.41-
7.33 (m, 3H), 7.11 (d, \textit{J} = 7.1 Hz, 1H), 6.85 (d, \textit{J} = 6.8 Hz, 1H), 5.09 (s, 2H), 4.72 (d, \textit{J} = 7.1 Hz,
2H), 3.89 (s, 3H), 2.02 (t, \textit{J} = 6.8 Hz, 1H); $^{13}$C NMR (CDCl$_3$, 125 MHz) $\delta$ 151.88, 147.42,
137.15, 132.84, 128.76, 128.71, 128.58, 125.96, 125.04, 112.95, 75.93, 58.27, 56.24; IR (neat)
3373 (bw), 3007 (bw), 2839 (w), 1579 (w), 1472 (s), 1439 (m), 1370 (m), 1272 (s), 1221 (s), 1080 (m), 1002 (bs), 802 (m), 745 (m), 695 (m) cm$^{-1}$; HRMS (ESI+) Calcd. for \ce{C15H15ClO3} [M]$^{+\cdot}$: 278.0710; Found 278.0695.
%***************[xbaw]%***************%

\vspace{10pt}
%***************[xbax]%***************%
\begin{wrapfigure}{l}{1.1in}
  \vspace{-25pt}
  \begin{center}
    \includegraphics[scale=0.8]{chp_singlecarbon/images/xbax}
    \begin{textblock}{1}(0.2,-0.7) \cmp{xbax} \end{textblock}
  \end{center}
  \vspace{-30pt}
\end{wrapfigure}\noindent \textbf{\CMPxbax}\\ (\ref{cmp:xbax}). Benzyl alcohol \ref{cmp:xbaw} (14.3
g, 51.3 mmol, 1.00 equiv) and CBr$_4$ (22.1 g, 66.7 mmol, 1.30 equiv) were weighed into a 250 mL
round bottom flask equipped with a magnetic stirbar and dissolved in 103 mL of THF. The solution was
cooled to 0 \degc\  and PPh$_3$ (17.5 g, 66.7 mmol, 1.30 equiv) was added as a solid. The reaction
mixture was then warmed to room temperature, and after 10 minutes diluted with water (50 mL),
poured into a separatory funnel, and the product was extracted with \ce{CH2Cl2} (3 x 50 mL). The
combined organics were dried over \ce{Na2SO4}, filtered, and concentrated. Purification by flash
column chromatography (30\% ethyl acetate in hexanes v/v) afforded the product \ref{cmp:xbax} as a
white solid (15.4 g, 87.7\%), mp 65-67 \degc. \\
R$_f$ = 0.25 (5\% Et$_2$O in pentane v/v); $^1$H NMR (CDCl$_3$, 500 MHz) $\delta$ 7.55-7.52 (m, 2H), 7.43-7.33 (m,
3H), 7.14 (d, \textit{J} = 9.0 Hz, 1H), 6.86 (d, \textit{J} = 9.0 Hz, 1H), 5.17 (s, 2H), 4.65 (s, 2H), 3.89 (s, 3H); $^{13}$C NMR (CDCl$_3$, 125 MHz) $\delta$ 151.94, 147.46, 137.29, 130.58, 128.67, 128.60, 128.56, 128.41,
126.36, 125.09, 113.49, 75.16, 56.24, 25.51; IR (neat) 3030 (bw), 2837 (bw), 1579 (w), 1474 (s),
1437 (m), 1372 (w), 1272 (s), 1236 (m), 1075 (m), 978 (bm), 802 (m), 696 (m) cm$^{-1}$; HRMS
(ESI+) Calcd. for \ce{C15H18BrClNO2} [M+NH$_4$]$^+$: 358.0209; Found 358.0212.
%***************[xbax]%***************%

\vspace{10pt}
%***************[xbay]%***************%
\begin{wrapfigure}{l}{1.1in}
  \vspace{-25pt}
  \begin{center}
    \includegraphics[scale=0.8]{chp_singlecarbon/images/xbay}
  \end{center}
  \vspace{-30pt}
\end{wrapfigure}\noindent \textbf{\CMPxbay}\ (\ref{cmp:xbay}). Benzyl bromide \ref{cmp:xbax} (34.1
g, 99.8 mmol, 1.00 equiv) was weighed into a 250 mL round bottom flask equipped with a magnetic
stirbar and dissolved in 166 mL of freshly distilled acetone. NaI (29.9 g, 198 mmol, 1.98 equiv) was then added as a solid, and the resulting suspension was stirred for 12 hours at room temperature in
the dark. The mixture was filtered through Celite$\textsuperscript{\textregistered}$ 545 rinsing with ethyl acetate (3 x 150 mL)
and concentrated. The crude residue was dissolved in 200 mL of ethyl acetate and poured into a
separatory funnel. The organics were washed with 50\% aqueous \ce{Na2S2O3} (150 mL), dried over
\ce{Na2SO4}, and concentrated to afford the desired product \ref{cmp:xbay} as a pale yellow solid
that was used without further purification (38.0 g, 98.1\%), mp 72-75 \degc. \\
$^1$H NMR (CDCl$_3$, 500 MHz) $\delta$ 7.56-7.52 (m, 2H), 7.43-7.34 (m, 3H), 7.08 (d, \textit{J} = 8.8 Hz, 1H),
6.83 (d, \textit{J} = 8.8 Hz, 1H), 5.22 (s, 2H), 4.55 (s, 2H), 3.88 (s, 3H); $^{13}$C NMR (CDCl$_3$, 125
MHz) $\delta$ 151.88, 146.77, 137.34, 131.75, 128.62, 128.41, 128.33, 125.87, 125.09, 112.84, 74.04,
56.18, $-$2.65; IR (neat) 3006 (bw), 2974 (bw), 2839 (w), 1575 (m), 1471 (s), 1460 (s), 1430 (s),
1366 (s), 1267 (s), 1223 (bs), 1099 (s), 1064 (s), 964 (s), 883 (m), 797 (s), 747 (s), 693 (s) cm$^{-1}$; HRMS (ESI+) Calcd. for \ce{C15H18ClINO2} [M+NH$_4$]$^+$: 406.0071; Found 406.0075.
%***************[xbay]%***************%

\vspace{10pt}
%***************[xbba]%***************%
\begin{wrapfigure}{l}{1.45in}
  \vspace{-25pt}
  \begin{center}
    \includegraphics[scale=0.8]{chp_singlecarbon/images/xbba}
  \end{center}
  \vspace{-30pt}
\end{wrapfigure}\noindent \textbf{\CMPxbba}\ (\ref{cmp:xbba}). In a drybox, NaH (37.9
mg, 1.58 mmol, 7.00 equiv) was weighed into a 2-neck, 10 mL round
bottom flask equipped with a magnetic stirbar. After removing the
flask from the drybox, a reflux condenser was installed. 1.7 mL of
DMSO was added, and the suspension was heated to 75 \degc\  for 1 hour. During this time, the
reaction became homogeneous, forming a teal-colored, clear solution. This solution was cooled to
room temperature, and a solution of \ce{Ph3PCH3I} (825 mg, 2.03 mmol, 9.00 equiv) in 2.8 mL of
DMSO was added over 30 minutes \textit{via} syringe pump. Upon addition of the salt, the reaction
mixture became bright yellow. After completion of the addition, the mixture was stirred for an
additional 30 minutes at room temperature, at which point a solution of racemic \ref{cmp:xbaz} (99.2
mg, 0.224 mmol, 1.00 equiv) in 0.62 mL of DMSO was added dropwise. The reaction mixture was then
heated to 75 \degc\  and stirred for 16 hours. The resulting amber solution was cooled to room
temperature and acidified by the addition of 5 mL of saturated aqueous NH$_4$Cl.
The reaction mixture was diluted with H$_2$O (15 mL), poured into a separatory funnel, and the
product was extracted with Et$_2$O (3 x 15 mL). The combined organics were washed with H$_2$O (15
mL), saturated aqueous NaCl (15 mL), dried over \ce{Na2SO4}, filtered, and concentrated.
Purification by flash column chromatography (10\% ethyl acetate in hexanes v/v) afforded the
unexpected compound \ref{cmp:xbba} as a white solid (59.5 mg, 60.2\%), mp 87-89 \degc. \\
R$_f$ = 0.26 (10\% ethyl acetate in hexanes v/v); $^1$H NMR (CDCl$_3$, 500 MHz) $\delta$ 6.86 (d,
\textit{J} = 8.6 Hz, 1H), 6.66 (d, J = 8.6 Hz, 1H), 4.80 (dd, \textit{J} = 1.7, 1.7 Hz, 1H), 4.76 (dd, \textit{J} = 2.2, 2.2 Hz, 1H), 4.04 (dd, \textit{J} = 6.1, 6.1 Hz, 1H), 3.84 (s, 3H), 2.98 (d, \textit{J} = 17.6 Hz, 1H), 2.47-2.37 (m, 2H), 2.29 (d, \textit{J} = 17.6 Hz, 1H),
1.90-1.82 (m, 1H), 1.81-1.73 (m, 3H), 1.69-1.57 (m, 3H), 1.19 (s, 3H), 1.00 (s, 3H); $^{13}$C NMR
(CDCl$_3$, 125 MHz) $\delta$ 161.44, 147.30, 143.75, 126.18, 120.11, 119.58, 109.90, 102.77, 79.33,
56.37, 53.18, 44.08, 34.39, 33.40, 31.80, 30.90, 29.50, 24.52, 24.35, 24.19; IR (neat) 3071 (w),
2951 (bm), 2916 (bm), 1649 (w), 1578 (m), 1476 (bs), 1308 (m), 1253 (m), 1230 (s), 1097 (m),
1051 (m), 799 (m), 782 (m), 673 (m) cm$^{-1}$; HRMS (ESI+) Calcd. for \ce{C20H26ClO4} [M+H]$^+$:
333.1621; Found 333.1619.
%***************[xbba]%***************%

\vspace{10pt}
%***************[xbbb]%***************%
\begin{wrapfigure}{l}{1.3in}
  \vspace{-25pt}
  \begin{center}
    \includegraphics[scale=0.8]{chp_singlecarbon/images/xbbb}
    \begin{textblock}{1}(2.8,-1) \cmp{xbbb} \end{textblock}
  \end{center}
  \vspace{-30pt}
\end{wrapfigure}\noindent \textbf{\CMPxbbb}\ (\ref{cmp:xbbb}).  To a solution of keto-
alcohol \ref{cmp:xbaz} (1.03 g, 2.32 mmol, 1.00 equiv) in 12 mL of DMF were
added imidazole (474 mg, 6.97 mmol, 3.00 equiv) and TBSCl (1.05 g, 6.97
mmol, 3.00 equiv) sequentially as solids. After stirring 5 hours at room
temperature, 5 mL of methanol was added and the reaction was stirred for
an additional 15 minutes. The reaction mixture was then poured into saturated NH$_4$Cl (20 mL)
and the product was extracted with Et$_2$O (5 x 10 mL). The combined organics were washed with
1 N HCl (30 mL), H$_2$O (30 mL), saturated aqueous NaCl (30 mL), dried over \ce{Na2SO4}, filtered,
and concentrated to afford the desired product \ref{cmp:xbbb} as a viscous oil that was used without
further purification (1.14 g, 88.4\%). \\
\rotation = $-$31.95 (c 0.87, CHCl$_3$); $^1$H NMR (CDCl$_3$, 500 MHz) $\delta$ 7.40-7.32 (m, 5H), 7.04 (d, J =
8.8 Hz, 1H), 6.76 (d, \textit{J} = 9.0 Hz, 1H), 5.02 (d, \textit{J} = 11.2 Hz, 1H), 4.91 (d, \textit{J} = 11.0 Hz, 1H), 3.83
(s, 3H), 3.66 (dd, \textit{J} = 7.3, 5.9 Hz, 1H), 3.42 (d, \textit{J} = 13.4 Hz, 1H), 2.86 (d, \textit{J} = 13.4 Hz, 1H), 2.61
(ddd, \textit{J} = 17.4, 5.9, 5.9 Hz, 1H), 2.07-1.97 (m, 2H), 1.84-1.77 (m, 1H), 1.77-1.67 (m, 2H), 1.52-
1.44 (m, 1H), 1.44-1.36 (m, 1H), 1.10 (s, 3H), 0.91 (s, 9H), 0.80 (s, 3H), 0.03-0.02 (m, 6H); $^{13}$C
NMR (CDCl$_3$, 125 MHz) $\delta$ 215.05, 151.19, 147.94, 137.56, 130.67, 128.79, 128.55, 128.32,
127.28, 124.39, 111.91, 80.62, 75.12, 56.89, 56.09, 52.03, 42.41, 37.09, 35.24, 32.13, 32.06,
27.20, 26.01, 25.26, 19.14, 18.24, $-$4.21, $-$4.77; IR (neat) 2956 (bs), 2876 (bm), 1705 (s), 1465
(bs), 1377 (bw), 1277 (s), 1214 (m), 1115 (bm), 1060 (s), 981 (bm), 836 (s), 775 (s), 698 (m)
cm$^{-1}$; HRMS (ESI+) Calcd. for \ce{C32H46ClO4Si} [M+H]$^+$: 557.2854; Found 557.2836.
%***************[xbbb]%***************%

\vspace{10pt}
%***************[xbbc]%***************%
\begin{wrapfigure}{l}{1.3in}
  \vspace{-25pt}
  \begin{center}
    \includegraphics[scale=0.8]{chp_singlecarbon/images/xbbc}
  \end{center}
  \vspace{-30pt}
\end{wrapfigure}\noindent \textbf{\CMPxbbc}\ (\ref{cmp:xbbc}). In a drybox, NaH (92.6
mg, 3.86 mmol, 7.00 equiv) was weighed into a 2-neck, 25 mL round
bottom flask equipped with a magnetic stirbar. After removing the flask
from the drybox, a reflux condenser was installed. 4.2 mL of DMSO was
added, and the suspension was heated to 75 \degc\  for 1 hour. During this
time, the reaction became homogeneous, forming a teal-colored, clear solution. This solution was
cooled to room temperature, and a solution of \ce{Ph3PCH3I} (2.01 mg, 4.96 mmol, 9.00 equiv) in 6.8
mL of DMSO was added over 30 minutes \textit{via} syringe pump. Upon addition of the salt solution,
the reaction mixture became bright yellow. After completion of the addition, the mixture was
stirred for an additional 30 minutes at room temperature, at which point a solution of ketone
\ref{cmp:xbbb} (307 mg, 0.551 mmol, 1.00 equiv) in 1.5 mL of DMSO and 1.5 mL of THF was added
dropwise.
The reaction mixture was then heated to 75 \degc\  and stirred for 16 hours. The resulting amber
solution was cooled to room temperature and acidified by the addition of 5 mL of saturated aqueous
NH$_4$Cl. The reaction mixture was diluted with H$_2$O (15 mL), poured into a separatory funnel, and
the product was extracted with Et$_2$O (3 x 20 mL). The combined organics were washed with H$_2$O
(25 mL), saturated aqueous NaCl (20 mL), dried over \ce{Na2SO4}, filtered, and concentrated.
Purification by flash column chromatography (20\% Et$_2$O in pentane v/v) afforded the
desired compound \ref{cmp:xbbc} as a white solid (296 mg, 96.7\%), mp 98-105 \degc. \\
\rotation = $+$15.37 (c 1.05, CHCl$_3$); R$_f$ = 0.33 (3\% Et$_2$O in pentane v/v); $^1$H NMR (CDCl$_3$, 500
MHz) $\delta$ 7.43-7.29 (m, 5H), 7.04 (d, \textit{J} = 8.8 Hz, 1H) 6.73 (d, \textit{J} = 8.8 Hz, 1H), 4.98-4.81 (m, 2H),
4.75-4.71 (m, 1H), 4.35-4.30 (m, 1H), 3.84 (s, 3H), 3.51 (dd, \textit{J} = 5.9, 2.0 Hz, 1H), 3.36 (d, J =
13.2 Hz, 1H), 2.76-2.65 (m, 1H), 2.73 (d, \textit{J} = 12.9 Hz, 1H), 2.05-1.93 (m, 2H), 1.93-1.84 (m,
1H), 1.77-1.68 (m, 1H), 1.43-1.34 (m, 1H), 1.30-1.20 (m, 1H), 1.20-1.09 (m, 2H), 1.14 (s, 3H),
0.92 (s, 9H), 0.82 (s, 3H), 0.04 (s, 3H), 0.03 (s, 3H); $^{13}$C NMR (CDCl$_3$, 125 MHz) $\delta$ 151.93,
151.25, 148.43, 137.74, 133.02, 128.62, 128.49, 128.10, 127.82, 124.31, 111.03, 110.56, 83.86,
75.02, 56.04, 55.78, 45.67, 43.81, 38.05, 33.26, 31.72, 30.03, 26.65, 26.12, 23.10, 22.71, 18.32, $-$4.26, $-$4.69; IR (neat) 2954 (bs), 2933 (bs), 2856 (bm), 1463 (s), 1438 (m), 1371 (bw), 1277 (bm),
1074 (bs), 1006 (m), 836 (s), 740 (m), 697 (m) cm$^{-1}$; HRMS (ESI+) Calcd. for
\ce{C33H48ClO3Si} [M+H]$^+$: 555.3061; Found 555.3084.
%***************[xbbc]%***************%

\vspace{10pt}
%***************[xbbd]%***************%
\begin{wrapfigure}{l}{1.3in}
  \vspace{-25pt}
  \begin{center}
    \includegraphics[scale=0.8]{chp_singlecarbon/images/xbbd}
    \begin{textblock}{1}(2.8,-1) \cmp{xbbd} \end{textblock}
  \end{center}
  \vspace{-30pt}
\end{wrapfigure}\noindent \textbf{\CMPxbbd}\ (\ref{cmp:xbbd}). To a solution of
\textit{tert}-butyldimethylsilyl ether \ref{cmp:xbbc} (1.27 g, 2.29 mmol, 1.00 equiv) in 5.7 mL of
THF was added \ce{TBAF.xH2O} (10.5 g, 37.5 mmol, 16.4 equiv) as a solid.
The resulting suspension was then sonicated (60 W) continuously at 50 \degc\ 
for 12 hours, during which time the reaction mixture became homogenous.
The reaction mixture was directly loaded onto a plug of silica gel and eluted with ethyl acetate to
afford the desired compound \ref{cmp:xbbd} as a solid that was used without further purification
(1.01 g, quantitative), mp 122-125 \degc. \\
\rotation = $+$34.99 (c 0.96, CHCl$_3$); R$_f$ = 0.48 (50\% ethyl acetate in hexanes v/v); $^1$H NMR (CDCl$_3$, 500
MHz) $\delta$ 7.42-7.30 (m, 5H), 7.04 (d, \textit{J} = 8.8 Hz, 1H), 6.73 (d, \textit{J} = 8.8 Hz, 1H), 4.97-4.85 (m, 2H),
4.77-4.74 (m, 1H), 4.37-4.33 (m, 1H), 3.84 (s, 3H), 3.56 (ddd, \textit{J} = 5.6, 4.2, 1.2 Hz, 1H), 3.33 (d, J
= 12.9 Hz, 1H), 2.75-2.63 (m, 1H), 2.73 (d, \textit{J} = 13.2 Hz, 1H), 2.04-1.92 (m, 3H), 1.81-1.72 (m,
1H), 1.46-1.38 (m, 1H), 1.37-1.27 (m, 2H), 1.20 (s, 3H), 1.19-1.12 (m, 1H), 0.82 (s, 3H); $^{13}$C
NMR (CDCl$_3$, 125 MHz) $\delta$ 151.56, 151.18, 148.45, 137.83, 132.73, 128.53, 128.47, 128.07,
127.77, 124.25, 111.06, 110.79, 83.99, 75.10, 56.02, 55.30, 45.32, 43.56, 37.89, 33.02, 30.71,
29.79, 26.45, 22.60, 21.87; IR (neat) 3377 (bw), 3058 (bw), 2917 (bw), 2848 (bw), 1647 (bw),
1479 (m), 1295 (bm), 1063 (bs), 925 (bm), 737 (s), 688 (bs) cm$^{-1}$; HRMS (ESI+) Calcd. for
\ce{C27H34ClO3} [M+H]$^+$: 441.2196; Found 441.2174.
%***************[xbbd]%***************%

\vspace{10pt}
%***************[xbbe]%***************%
\begin{wrapfigure}{l}{1.2in}
  \vspace{-25pt}
  \begin{center}
    \includegraphics[scale=0.8]{chp_singlecarbon/images/xbbe}
  \end{center}
  \vspace{-30pt}
\end{wrapfigure}\noindent \textbf{\CMPxbbe}\ (\ref{cmp:xbbe}). Alcohol \ref{cmp:xbbd} (1.01 g, 2.29
mmol, 1.00 equiv) was weighed into a 50 mL round bottom flask equipped with a
magnetic stirbar and dissolved in 23 mL of wet \ce{CH2Cl2}. The solution was
cooled to 4 \degc\  and DMP (2.91 g, 6.87 mmol, 3.00 equiv)
was added as a solid. The reaction mixture was stirred for 12 hours at 4 \degc,
at which point additional DMP (2.02 g, 4.58 mmol, 2.00 equiv) and 50 $\mu$L of
H$_2$O were added. The reaction was warmed to room temperature and stirred for an additional
hour. The reaction mixture was poured into 1 N NaOH (100 mL), and the product was extracted
with \ce{CH2Cl2} (3 x 25 mL). The combined organics were dried over \ce{Na2SO4}, filtered, and
concentrated. Purification by column chromatography (20\% ethyl acetate in hexanes v/v) afforded the
desired compound \ref{cmp:xbbe} as a white foam (1.00 g, quantitative), mp 95-98 \degc. \\
\rotation = $+$11.12 (c 1.17, CHCl$_3$); R$_f$ = 0.27 (20\% ethyl acetate in hexanes v/v); $^1$H NMR (CDCl$_3$, 500
MHz) $\delta$ 7.40-7.30 (m, 5H), 7.06 (d, \textit{J} = 8.8 Hz, 1H), 6.76 (d, \textit{J} = 8.8 Hz, 1H), 4.98-4.86 (m, 2H),
4.80-4.77 (m, 1H), 4.43-4.39 (m, 1H), 3.86 (s, 3H), 3.34 (d, \textit{J} = 12.9 Hz, 1H), 2.74 (d, \textit{J} = 12.9
Hz, 1H), 2.71-2.60 (m, 1H), 2.38 (dd, \textit{J} = 19.3, 8.5 Hz, 1H), 2.11-1.96 (m, 2H), 1.92-1.81 (m, 2H), 1.52-1.42 (m, 1H), 1.35 (ddd, \textit{J} = 13.4, 13.4, 3.9 Hz, 1H), 1.23 (s, 3H), 1.20-1.12 (m, 1H),
0.89 (s, 3H); $^{13}$C NMR (CDCl$_3$, 125 MHz) $\delta$ 222.21, 151.19, 150.50, 148.50, 137.64, 132.08,
128.64, 128.54, 128.27, 127.63, 124.34, 111.32, 111.24, 75.30, 57.35, 56.03, 48.82, 43.20, 38.17,
35.51, 30.61, 29.22, 21.81, 21.58, 21.49; IR (neat) 2935 (bm), 2856 (bw), 1734 (s), 1575 (w),
1464 (bs), 1406 (m), 1277 (s), 1234 (bm), 1072 (m), 978 (bm), 896 (bm), 798 (m), 698 (m) cm$^{-1}$;
HRMS (ESI+) Calcd. for \ce{C27H32ClO3} [M+H]$^+$: 439.2040; Found 439.2024.
%***************[xbbe]%***************%

\vspace{10pt}
%***************[xbbf]%***************%
\begin{wrapfigure}{l}{1.2in}
  \vspace{-25pt}
  \begin{center}
    \includegraphics[scale=0.8]{chp_singlecarbon/images/xbbf}
    \begin{textblock}{1}(2.5,-1) \cmp{xbbf} \end{textblock}
  \end{center}
  \vspace{-30pt}
\end{wrapfigure}\noindent \textbf{\CMPxbbf}\ (\ref{cmp:xbbf}). Silyl ether-alkene \ref{cmp:xbbc}
(263 mg, 0.473 mmol, 1.00 equiv) and \ce{RhCl3.H2O} (14.7 mg, 0.0702 mmol, 0.148 equiv) were weighed
into 5 mL 2-neck round bottom flask equipped with a magnetic stirbar and a reflux condenser and dissolved in
1.2 mL of EtOH and 1.2 mL of CHCl$_3$. The resulting deep red solution then
was heated to 55 \degc\  for 15 hours. The reaction mixture was then cooled to room temperature and
concentrated. The crude residue was purified by column chromatography (25\% ethyl acetate in
hexanes) to afford the desired compound \ref{cmp:xbbf} as a clear oil that was used directly in the
next step (208 mg, quantitative).\\
\rotation = $-$103.49 (c 0.85, CHCl$_3$); R$_f$ = 0.38 (30\% ethyl acetate in hexanes v/v); $^1$H NMR (CDCl$_3$,
500 MHz) $\delta$ 7.45-7.41 (m, 2H), 7.39-7.30 (m, 3H), 7.07 (d, \textit{J} = 8.8 Hz, 1H), 6.75 (d, \textit{J} = 8.8 Hz,
1H), 5.35-5.32 (m, 1H), 4.94 (d, \textit{J} = 11.0 Hz, 1H), 4.88 (d, \textit{J} = 11.0 Hz, 1H), 3.86 (s, 3H), 3.52
(ddd, \textit{J} = 6.1, 6.1, 0 Hz, 1H), 3.24 (d, \textit{J} = 12.9 Hz, 1H), 2.75 (d, \textit{J} = 12.9 Hz, 1H), 2.07-2.01 (m,
1H), 1.92-1.85 (m, 1H), 1.84-1.76 (m, 2H), 1.67 (dddd, \textit{J} = 15.3, 7.6, 7.6, 2.0 Hz, 1H), 1.47-1.43
(m, 3H), 1.39-1.30 (m, 2H), 1.14 (s, 3H), 1.11-1.02 (m, 1H), 0.87 (s, 3H); $^{13}$C NMR (CDCl$_3$, 125
MHz) $\delta$ 151.49, 148.35, 140.37, 137.71, 133.32, 128.45, 128.40, 128.09, 127.78, 124.63, 121.62,
111.00, 82.91, 74.81, 56.01, 55.17, 42.96, 42.51, 36.88, 34.60, 31.52, 26.91, 24.69, 22.15, 21.29;
IR (neat) 3389 (bw), 3064 (bw), 2955 (bm), 2873 (bm), 1574 (w), 1464 (bs), 1373 (m), 1276 (s), 1178 (m), 1074 (bm), 981 (bm), 797 (m), 697 (m) cm$^{-1}$; HRMS (ESI+) Calcd. for
\ce{C27H33ClO3} [M+H]$^+$: 441.2196; Found 441.2182.
%***************[xbbf]%***************%

\pagebreak
%***************[xbbg]%***************%
\begin{wrapfigure}{l}{1.25in}
  \vspace{-15pt}
  \begin{center}
    \includegraphics[scale=0.8]{chp_singlecarbon/images/xbbg}
  \end{center}
  \vspace{-30pt}
\end{wrapfigure}\noindent \textbf{\CMPxbbg}\ (\ref{cmp:xbbg}). Alcohol \ref{cmp:xbbf} (208 mg, 0.473
mmol, 1.00 equiv) was weighed into a 10 mL round bottom flask equipped with a
magnetic stirbar and dissolved in 4.7 mL of wet \ce{CH2Cl2}. DMP (602 mg, 1.42 mmol, 3.00 equiv) was
then added as a solid and the reaction mixture was stirred for 1.5 hours at room temperature. The
reaction mixture was poured into 1 N NaOH (20 mL), and the product was extracted with \ce{CH2Cl2}
(3 x 10 mL). The combined organics were dried over \ce{Na2SO4}, filtered, and concentrated.
Purification by column chromatography (12\% ethyl acetate in hexanes v/v) afforded the desired
compound \ref{cmp:xbbg} as a colorless oil (203 mg, 97.8\%, 2 steps).\\
\rotation = $-$84.42 (c 0.95, CHCl$_3$); R$_f$ = 0.33 (15\% ethyl acetate in hexanes v/v); $^1$H NMR (CDCl$_3$,
500 MHz) $\delta$ 7.40-7.30 (m, 5h), 7.09 (d, \textit{J} = 8.8 Hz, 1H), 6.77 (d, \textit{J} = 8.8 Hz, 1H), 5.29-5.26 (m,
1H), 4.96 (d, \textit{J} = 11.0 Hz, 1H), 4.90 (d, \textit{J} = 11.2 Hz, 1H), 3.87 (s, 3H), 3.20 (d, \textit{J} = 13.2, Hz, 1H),
2.77 (d, \textit{J} = 13.0 Hz, 1H), 2.32 (dd, \textit{J} = 18.6, 7.6 Hz, 1H), 2.18 (dd, \textit{J} = 11.7, 6.4 Hz, 1H), 2.13-
2.05 (m, 1H), 2.05-1.98 (m, 1H), 1.91-1.83 (m, 1H), 1.72 (dddd, \textit{J} = 18.1, 2.0, 2.0, 2.0 Hz, 1H),
1.50-1.45 (m, 3H), 1.34 (dddd, \textit{J} = 12.2, 12.2, 12.2, 8.5 Hz, 1H), 1.25 (s, 3H), 0.94 (s, 3H); 13C
NMR (CDCl$_3$, 125 MHz) $\delta$ 223.13, 151.52, 148.33, 139.33, 137.56, 132.77, 128.51, 128.40,
128.22, 127.75, 124.73, 119.92, 111.18, 74.92, 56.03, 54.11, 47.07, 41.61, 37.68, 36.04, 30.82,
25.03, 23.58, 21.89, 21.05; IR (neat) 2966 (bm), 2935 (bw), 1736 (s), 1464 (bs), 1372 (m), 1277
(s), 1242 (bm), 1076 (m), 984 (bm), 798 (m), 698 (m) cm$^{-1}$; HRMS (ESI+) Calcd. for
\ce{C27H34ClO3} [M+H]$^+$: 439.2040; Found 439.2037.
%***************[xbbg]%***************%

\vspace{10pt}
%***************[xbbh]%***************%
\begin{wrapfigure}{l}{1.3in}
  \vspace{-25pt}
  \begin{center}
    \includegraphics[scale=0.8]{chp_singlecarbon/images/xbbh}
  \end{center}
  \vspace{-30pt}
\end{wrapfigure}\noindent \textbf{\CMPxbbh}\ (\ref{cmp:xbbh}). In a drybox, Sc(OTf)$_3$ (4.2
mg, 0.0086 mmol, 0.050 equiv) was weighed directly into a J. Young NMR
tube. A solution of ketone \ref{cmp:xbbe} (75.2 mg, 0.171 mmol, 1.00 equiv) in
0.48 mL of CDCl$_3$ was transferred directly to the solid Sc(OTf)$_3$. The
cloudy gray suspension was allowed to stand for 15 minutes at which point
TMSD (174 $\mu$L, 0.342 mmol, 2.00 equiv, 2.47 M in hexanes) was introduced dropwise. The
reaction tube was removed from the drybox, connected to a nitrogen manifold, and allowed to
stand at room temperature for 12 hours. The reaction mixture was then warmed to 50 \degc\  for 48
hours. $^1$H NMR analysis indicated roughly 98\% conversion and an approximate 5:1 ratio
of regioisomeric silyl products. The reaction mixture was poured into H$_2$O (5 mL), and the products
were extracted with Et$_2$O (20 mL). The organics were washed with saturated aqueous NaCl (10
mL), dried over \ce{Na2SO4}, filtered, and concentrated. The enol-silane products were then purified
away from a trace amount of starting material by column chromatography (7\% ethyl acetate in hexanes
v/v). The purified product mixture was then dissolved in 2 mL of THF, \ce{TBAF.xH2O} (95.8 mg, 0.342
mmol, 2.00 equiv) was added as a solid, and the reaction mixture was allowed to stir for 10 minutes at room temperature. The solution
was concentrated and purified by column chromatography (15 to 25\% ethyl acetate in hexanes v/v) to
afford the desired homologated ketone \ref{cmp:xbbh} as a white solid (53.4 mg, 68.8\%), mp 88-92
\degc. \\
\rotation = $+$8.23 (c 0.65, CHCl$_3$); R$_f$ = 0.34 (15\% ethyl acetate in hexanes v/v); $^1$H NMR (CDCl$_3$, 500
MHz) $\delta$ 7.40-7.30 (m, 5H), 7.05 (d, \textit{J} = 8.8 Hz, 1H), 6.75 (d, \textit{J} = 8.8 Hz, 1H), 4.97 (d, \textit{J} = 11.0
Hz, 1H), 4.88 (d, \textit{J} = 11.0 Hz, 1H), 4.77-4.74 (m, 1H), 4.46-4.43 (m, 1H), 3.86 (s, 3H), 3.37 (d, J
= 12.9 Hz, 1H), 2.82 (d, \textit{J} = 12.9 Hz, 1H), 2.63 (ddd, \textit{J} = 13.9, 4.4, 4.4 Hz, 1H), 2.45-2.37 (m,
1H), 2.33-2.26 (m, 1H), 2.12 (ddd, \textit{J} = 14.6, 5.1, 5.1 Hz, 1H), 1.95-1.76 (m, 4H), 1.59-1.49 (m,
1H), 1.38-1.31 (m, 2H), 1.33 (s, 3H), 0.90 (s, 3H); $^{13}$C NMR (CDCl$_3$, 125 MHz) $\delta$ 216.60, 151.80, 151.31, 148.28, 137.69, 132.98, 128.55, 128.25, 127.59, 124.43, 111.07, 110.30, 75.16,
56.00, 55.82, 50.90, 44.72, 39.92, 37.07, 32.57, 29.70, 24.24, 24.16, 22.82; IR (neat) 3087 (bw),
2938 (bm), 2861 (bm), 1698 (s), 1575 (w), 1462 (s), 1438 (m), 1372 (m), 1276 (s), 1214 (bm),
980 (bm), 798 (m), 698 (m) cm$^{-1}$; HRMS (ESI+) Calcd. for \ce{C28H34ClO3} [M+H]$^+$: 453.2196;
Found 453.2209.
%***************[xbbh]%***************%

\vspace{10pt}
%***************[xbbi]%***************%
\begin{wrapfigure}{l}{1.3in}
  \vspace{-15pt}
  \begin{center}
    \includegraphics[scale=0.8]{chp_singlecarbon/images/xbbi}
  \end{center}
  \vspace{-30pt}
\end{wrapfigure}\noindent \textbf{\CMPxbbi}\ (\ref{cmp:xbbi}). The minor regioisomer \ref{cmp:xbbi}
was isolated from the reaction above as a colorless oil (6.5 mg, 8.4\%). \\
\rotation = $+$25.56 (c 0.59, CHCl$_3$); R$_f$ = 0.25 (15\% ethyl acetate in
hexanes v/v); $^1$H NMR (CDCl$_3$, 500 MHz) $\delta$ 7.42-7.32 (m, 5H), 7.6 (d, \textit{J} =
8.8 Hz, 1H), 6.76 (d, \textit{J} = 8.8 Hz, 1H), 5.01 (d, \textit{J} = 10.2, 1H), 5.01 (d, \textit{J} =
10.2 Hz, 1H), 4.89 (d, \textit{J} = 11.2 Hz, 1H), 4.76-4.72 (m, 1H), 4.43-4.39 (m, 1H), 3.86 (s, 3H), 3.38
(d, \textit{J} = 12.9 Hz, 1H), 2.88 (d, \textit{J} = 13.2 Hz, 1H), 2.67-2.58 (m, 1H), 2.24 (d, \textit{J} = 13.9, 1H), 2.24-
2.18 (m, 2H), 2.08 (ddd, \textit{J} = 14.6, 4.4, 4.4 Hz, 1H), 2.03-1.96 (m, 1H), 1.97 (d, \textit{J} = 13.7, 1H),
1.76-1.69 (m, 1H), 1.52-1.44 (m, 2H), 1.31-1.24 (m, 1H), 1.18 (s, 3H), 0.90 (s, 3H); $^{13}$C NMR
(CDCl$_3$, 125 MHz) $\delta$ 212.83, 151.32, 151.06, 148.34, 137.92, 132.92, 128.55, 128.48, 128.19,
127.63, 124.42, 111.14, 110.74, 75.12, 56.20, 56.05, 53.03, 44.95, 40.33, 40.24, 39.50, 34.10,
31.46, 29.98, 26.03, 23.18; IR (neat) 3086 (bw), 2936 (bm), 2853 (bm), 1716 (s), 1464 (s), 1438
(bm), 1373 (bw), 1275 (s), 1215 (bm), 1102 (bm), 985 (bm), 798 (m), 698 (m) cm$^{-1}$;
HRMS (ESI+) Calcd. for \ce{C28H34ClO3} [M+H]$^+$: 453.2196; Found 453.2218.
%***************[xbbi]%***************%

\vspace{10pt}
%***************[xbbj]%***************%
\begin{wrapfigure}{l}{1.2in}
  \vspace{-25pt}
  \begin{center}
    \includegraphics[scale=0.8]{chp_singlecarbon/images/xbbj}
  \end{center}
  \vspace{-30pt}
\end{wrapfigure}\noindent \textbf{\CMPxbbj}\ (\ref{cmp:xbbj}). In a drybox, Sc(OTf)$_3$ (6.5 mg,
0.015 mmol, 0.045 equiv) was weighed directly into a 1.5 mL vial
equipped with a magnetic stirbar. A solution of ketone \ref{cmp:xbbg} (128 mg,
0.292 mmol, 1.00 equiv) in CDCl$_3$ (0.53 mL) was transferred directly to the solid Sc(OTf)$_3$. The
cloudy gray suspension was stirred for 15 minutes at which point TMSD (236 $\mu$L, 0.583 mmol,
2.00 equiv, 2.47 M in hexanes) was introduced dropwise. The entire reaction mixture (including
any residual solids) was transferred \textit{via} glass pipette to a J. Young NMR tube, and the vial
was rinsed with an additional 0.2 mL of CDCl$_3$. The reaction tube was removed from the drybox,
connected to a nitrogen manifold, and placed in an oil bath pre-heated to 50 \degc. After 16 hours
of heating, the reaction was cooled to room temperature. $^1$H NMR analysis indicated complete
conversion. The reaction mixture was poured into H$_2$O (5 mL), and the product was extracted
with Et$_2$O (20 mL). The organics were washed with saturate aqueous NaCl (10 mL), dried over
\ce{Na2SO4}, filtered, and concentrated. The crude residue was then dissolved in 2 mL of THF,
\ce{TBAF.xH2O} (164 mg, 0.584 mmol, 2.00 equiv) was added as a solid, and the reaction mixture was
allowed to stir for 10 minutes at 23 \degc. The solution was concentrated and purified by column
chromatography (15\% ethyl acetate in hexanes v/v) to afford the desired homologated ketone
\ref{cmp:xbbj} as a colorless oil (124 mg, 93.4\%).\\
\rotation = $-$32.35 (c 0.83, CHCl$_3$); R$_f$ = 0.57 (30\% ethyl acetate in hexanes v/v); $^1$H NMR (CDCl$_3$,
500 MHz) $\delta$ 7.42-7.39 (m, 2H), 7.38-7.30 (m, 3H), 7.09 (d, \textit{J} = 8.8 Hz, 1H), 6.77 (d, \textit{J} = 8.8 Hz,
1H), 5.36-5.32 (m, 1H), 4.96 (d, \textit{J} = 10.7 Hz, 1H), 4.88 (d, \textit{J} = 10.7 Hz, 1H), 3.88 (s, 3H), 3.14
(d, \textit{J} = 13.7 Hz, 1H), 2.99 (d, \textit{J} = 13.9 Hz, 1H), 2.50 (ddd, \textit{J} = 14.6, 12.6, 6.8, 1H), 2.46-2.40 (m,
1H), 2.39-2.35 (m, 1H), 2.27-2.21 (m, 1H), 1.93-1.86 (m, 1H), 1.78-1.72 (m, 1H), 1.68-1.66 (m,
3H), 1.64-1.58 (m, 1H), 1.35 (s, 3H), 1.34-1.18 (m, 2H), 0.82 (s, 3H); $^{13}$C NMR (CDCl$_3$, 125
MHz) $\delta$ 216.42, 151.57, 148.21, 139.89, 137.28, 133.52, 128.72, 128.56, 128.29, 127.58, 124.82,
119.27, 111.02, 74.83, 55.97, 51.23, 49.70, 42.12, 37.94, 37.23, 32.88, 24.73, 24.47, 23.74,
20.49; IR (neat) 3030 (bw), 2955 (bm), 2861 (bm), 1698 (s), 1574 (m), 1462 (bs), 1371 (m), 1276
(s), 1214 (bm), 1080 (bs), 979 (bs), 924 (bm), 797 (s), 732 (bs), 697 (s) cm$^{-1}$; HRMS (ESI+)
Calcd. for \ce{C28H34ClO3} [M+H]$^+$: 453.2196; Found 453.2210.
%***************[xbbj]%***************%

\vspace{10pt}
%***************[xbbk]%***************%
\begin{wrapfigure}{l}{1.0in}
  \vspace{-25pt}
  \begin{center}
    \includegraphics[scale=0.8]{chp_singlecarbon/images/xbbk}
    \begin{textblock}{1}(0.2,-0.7) \cmp{xbbk} \end{textblock}
  \end{center}
  \vspace{-30pt}
\end{wrapfigure}\noindent \textbf{\CMPxbbk}\ (\ref{cmp:xbbk}). Benzyl protected \textit{o}-
vanillin\footnote{Prepared in a single step from \textit{o}-vanillin according to the literature
method. See reference \ref{ref:holz} for details.} (3.40 g, 14.0 mmol, 1.00 equiv) was weighed into
a 50 mL round bottom flask equipped with a magnetic stirbar and dissolved in 14 mL of CH$_3$CN.
\ce{NaH2PO4} (5.6 mL of a 0.67 M solution in H$_2$O, 3.80 mmol, 0.271 equiv) was then added followed
by \ce{H2O2} (1.4 mL of a 30\% wt/wt solution in H$_2$O, 14.7 mmol,
1.05 equiv). The reaction flask was placed in a water bath, and NaClO$_2$ (2.37 g, 21.0 mmol, 1.50
equiv) in 21 mL of water was added dropwise over 2 hours. Upon completion of the addition, the
reaction was allowed to stir, open to the air, for 12 hours at room temperature. The reaction
mixture was poured into 1 N HCl (100 mL) and transferred to a separatory funnel. The product
was extracted with \ce{CH2Cl2} (3 x 50 mL) and the organics were washed with 50\% aqueous
\ce{Na2S2O3} (200 mL). The aqueous phase was back-extracted with additional \ce{CH2Cl2} (2 x 100
mL), and the combined organics were dried over MgSO$_4$, filtered, and concentrated to afford the
desired product \ref{cmp:xbbk} as white solid that was used without further purification (3.34 g,
92.3\%), mp 81-83 \degc.\\
$^1$H NMR (CDCl$_3$, 500 MHz) $\delta$ 11.40 (s, 1H), 7.70 (dd, \textit{J} = 7.6, 2.0 Hz, 1 H), 7.45-7.37 (m, 5H),
7.21-7.15 (m, 2H), 5.27 (s, 2H), 3.96 (s, 3H); $^{13}$C NMR (DMSO-\textit{d}$_6$, 125 MHz) $\delta$ 167.50, 153.25,
146.29, 137.60, 128.16, 128.03, 127.80, 127.78, 124.21, 121.22, 115.72, 74.59, 56.02; IR (neat)
3017 (bw), 1697 (bs), 1579 (m), 1471 (m), 1459 (m), 1312 (s), 1287 (m), 1260 (s), 1204 (s), 1172
(m), 1089 (m), 1052 (s), 974 (s), 866 (m), 750 (bs), 697 (s) cm$^{-1}$; HRMS (ESI+) Calcd. for
\ce{C15H15O4} [M+H]$^+$: 259.0970; Found 259.0969.
%***************[xbbk]%***************%

\vspace{10pt}
%***************[xbbl]%***************%
\begin{wrapfigure}{l}{1.1in}
  \vspace{-25pt}
  \begin{center}
    \includegraphics[scale=0.8]{chp_singlecarbon/images/xbbl}
    \begin{textblock}{1}(0.2,-0.7) \cmp{xbbl} \end{textblock}
  \end{center}
  \vspace{-30pt}
\end{wrapfigure}\noindent \textbf{\CMPxbbl}\ (\ref{cmp:xbbl}). (a) In a drybox, LiAlD$_4$ (605 mg,
14.4 mmol, 1.00 equiv) was weighed into a 250 mL round bottom flask equipped with a magnetic
stirbar. After removing the flask from the drybox, 100 mL of THF was added, and the resulting grey
suspension was cooled to 0 \degc. In a separate flask, acid \ref{cmp:xbbk} (3.73 g, 14.4 mmol. 1.00
equiv) was suspended in 44 mL of THF. The slurry of \ref{cmp:xbbk} was added to the LiAlD$_4$
suspension via cannula, and the reaction mixture was allowed to warm slowly to room temperature and stir for 20 hours. The dark grey solution was
then re-cooled to 0 \degc, and H$_2$O (50 mL) was slowly added to quench the excess LiAlD$_4$. The
resulting thick slurry was warmed to room temperature and diluted with 100 mL of 1 N HCl. The
reaction mixture was poured into a separatory funnel and the product was extracted with Et$_2$O (3
x 100 mL). The combined organics were washed with saturated aqueous NaHCO$_3$ (150 mL),
saturated aqueous NaCl (250 mL), dried over \ce{Na2SO4}, filtered, and concentrated.
(b) The crude solid was dissolved in 29 mL of \ce{CH2Cl2} and the solution was cooled to 4 \degc.
1,3- dichloro-5,5-dimethylhydantoin (3.41 g, 17.3 mmol, 1.20 equiv) was added as a solid, and the
reaction mixture was then stirred for 20 hours at 4 \degc. The resulting slurry was diluted with
saturated aqueous \ce{Na2S2O3} (20 mL), and the product was extracted with \ce{CH2Cl2} (3 x 10 mL).
The combined organics were washed with saturated aqueous NaHCO$_3$ (30 mL), H$_2$O (30 mL),
saturated aqueous NaCl (30 mL), dried over Na$_2$SO$_4$, filtered, and concentrated. The crude
residue was purified by column chromatography (30\% ethyl acetate in hexanes v/v) to afford the
desired compound \ref{cmp:xbbl} as a white solid (3.55 g, 85.1\%, 2 steps), mp 81-83 \degc.\\
$^1$H NMR (CDCl$_3$, 500 MHz) $\delta$ 7.47-7.43 (m, 2H), 7.42-7.33 (m, 3H), 7.11 (d, \textit{J} = 8.8 Hz, 1H),
6.85 (d, \textit{J} = 8.8 Hz, 1H), 5.09 (s, 2H), 3.89 (s, 3H), 2.06 (s, 1H);
$^{13}$C NMR (CDCl$_3$, 125 MHz) $\delta$ 151.81, 147.35, 137.10, 132.67, 128.70, 128.66, 128.51, 125.90, 124.98, 112.88, 75.87,
56.17; IR (neat) 3364 (bm), 3034 (w), 2943 (bw), 1576 (m), 1468 (bs), 1439 (s), 1368 (s), 1299
(m), 1236 (bs), 1082 (m), 1046 (s), 964 (bs), 844 (m), 801 (bs), 691 (s) cm$^{-1}$; HRMS (ESI+)
Calcd. for \ce{C15H17D2ClNO3} [M+NH$_4$]$^+$: 298.1179; Found 298.1177.
%***************[xbbl]%***************%

\vspace{10pt}
%***************[xbbm]%***************%
\begin{wrapfigure}{l}{1.1in}
  \vspace{-25pt}
  \begin{center}
    \includegraphics[scale=0.8]{chp_singlecarbon/images/xbbm}
    \begin{textblock}{1}(0.2,-0.7) \cmp{xbbm} \end{textblock}
  \end{center}
  \vspace{-30pt}
\end{wrapfigure}\noindent \textbf{\CMPxbbm}\\ (\ref{cmp:xbbm}). Benzyl alcohol \ref{cmp:xbbl} (2.98
g, 10.6 mmol, 1.00 equiv) and CBr$_4$ (4.57 g, 13.8 mmol, 1.30 equiv) were weighed into a 50 mL
round bottom flask equipped with a magnetic stirbar and dissolved in 21 mL of THF. The
solution was cooled to 0 \degc\  and PPh$_3$ (3.62 g, 13.8 mmol, 1.30 equiv) was added as a solid.
The reaction mixture was then warmed to room temperature, and after 10 minutes diluted with water
(20 mL), poured into a separatory funnel, and the product was extracted with \ce{CH2Cl2} (3 x 10
mL). The combined organics were dried over \ce{Na2SO4}, filtered, and concentrated. Purification by
flash column chromatography (30\% ethyl acetate in hexanes v/v) afforded the product \ref{cmp:xbbm}
as a white solid (3.40 g, 93.4\%), mp 62-65 \degc.\\
$^1$H NMR (CDCl$_3$, 500 MHz) $\delta$ 7.55-7.51 (m, 2H), 7.43-7.34 (m, 3H), 7.12 (d, \textit{J} = 8.8 Hz, 1H),
6.86 (d, \textit{J} = 8.8 Hz, 1H), 5.16 (s, 2H), 3.89 (s, 3H); $^{13}$C NMR (CDCl$_3$, 125 MHz) $\delta$ 151.90,
147.42, 137.27, 130.42, 128.64, 128.53, 128.39, 126.29, 125.06, 113.47, 75.13, 56.21; IR (neat)
3031 (w), 2944 (bw), 1577 (m), 1468 (s), 1438 (m), 1267 (s), 1237 (s), 1097 (s), 971 (s), 948 (s),
918 (m), 892 (m), 804 (s), 765 (s), 691 (s), 572 (m) cm$^{-1}$; HRMS (ESI+) Calcd. for
\ce{C15H16D2BrClNO2} [M+NH$_4$]$^+$: 360.0335; Found 360.0334.
%***************[xbbm]%***************%

\vspace{10pt}
%***************[xbbn]%***************%
\begin{wrapfigure}{l}{1.1in}
  \vspace{-25pt}
  \begin{center}
    \includegraphics[scale=0.8]{chp_singlecarbon/images/xbbn}
    \begin{textblock}{1}(0.2,-0.7) \cmp{xbbn} \end{textblock}
  \end{center}
  \vspace{-30pt}
\end{wrapfigure}\noindent \textbf{\CMPxbbn}\\ (\ref{cmp:xbbn}). Benzyl bromide \ref{cmp:xbbm} (3.06
g, 8.90 mmol, 1.00 equiv) was weighed into a 50 mL round bottom flask equipped with a magnetic
stirbar and dissolved in 15 mL of freshly distilled acetone. NaI (2.67 g, 17.8 mmol, 2.00
equiv) was then added as a solid, and the resulting suspension was stirred for 12 hours at room
temperature in the dark. The mixture was filtered through Celite$\textsuperscript{\textregistered}$ 545 rinsing with ethyl acetate (3 x 10 mL) and concentrated. The crude residue was dissolved in 30 mL of ethyl acetate and
poured into a separatory funnel. The organics were washed with 50\% aqueous \ce{NaS2O3} (20 mL),
dried over \ce{Na2SO4}, and concentrated to afford the desired product \ref{cmp:xbbn} as a pale
yellow solid that was used without further purification (3.44 g, 98.8\%), mp 74-76 \degc. \\
$^1$H NMR (CDCl$_3$, 500 MHz) $\delta$ 7.56-7.53 (m, 2H), 7.43-7.34 (m, 3H), 7.08 (d, \textit{J} = 8.8 Hz, 1H),
6.83 (d, \textit{J} = 8.8 Hz, 1H), 5.22 (s, 2H), 3.88 (s, 3H);
$^{13}$C NMR (CDCl$_3$, 125 MHz) $\delta$ 151.88,
146.78, 137.35, 131.66, 128.64, 128.42, 128.34, 125.86, 125.10, 112.86, 74.05, 56.19; IR (neat)
3005 (bw), 2837 (bw), 1573 (m), 1467 (s), 1437 (m), 1367 (m), 1296 (m), 1266 (s), 1235 (s),
1095 (s), 971 (bs), 877 (m), 802 (s), 744 (s), 692 (s) cm$^{-1}$; HRMS (ESI+) Calcd. for
\ce{C15H16D2ClINO2} [M+NH$_4$]$^+$: 408.0196; Found 408.0182.
%***************[xbbn]%***************%


%%%%%%%%%%%%%%%%%%%%%%%%%%%%%%%%%%%%%%%%%%%%%%%%%%%%%%%%%%%%%%%%%%%%%%%%%%%%%%%%%%%%%%%%
% End of experimental procedures for single carbon homologation chapter.
%%%%%%%%%%%%%%%%%%%%%%%%%%%%%%%%%%%%%%%%%%%%%%%%%%%%%%%%%%%%%%%%%%%%%%%%%%%%%%%%%%%%%%%%
\pagebreak
\subsection{NMR Spectral Data}

%=-=-=-=-=-=-=-=-=-=-=-=-=-=-=-=-=-=-=-=-=-=-=-=-=-=-=-=-=-=-=-=-=-=-=-=-=-=-=-=-=
% Single Carbon Homologation Compounds
%=-=-=-=-=-=-=-=-=-=-=-=-=-=-=-=-=-=-=-=-=-=-=-=-=-=-=-=-=-=-=-=-=-=-=-=-=-=-=-=-=

%=[xbaa]=-=-=-=-=-=-=-=-=-=-=-=-=-=-=-=-=-=-=-=-=-=-=-=-=-=-=-=-=-=-=-=-=-=-=-=-=-=-=
\begin{textblock}{20}(0,0)
\begin{figure}[htb]
\caption{$^1$H NMR of \CMPxbaa\ (\ref{cmp:xbaa})}
\includegraphics[scale=0.75, trim = 0mm 0mm 0mm 5mm,
clip]{chp_singlecarbon/images/nmr/xbaaH}
\vspace{-100pt}
\end{figure}
\end{textblock}
\begin{textblock}{1}(2,4)
\includegraphics[scale=0.8, angle=90]{chp_singlecarbon/images/xbaa}
\end{textblock}
\clearpage
%%%
\begin{textblock}{20}(0,0)
\begin{figure}[htb]
\caption{$^{13}$C NMR of  \CMPxbaa\ (\ref{cmp:xbaa})}
\includegraphics[scale=0.75, trim = 0mm 0mm 0mm 5mm,
clip]{chp_singlecarbon/images/nmr/xbaaC}
\vspace{-100pt}
\end{figure}
\end{textblock}
\begin{textblock}{1}(2,2)
\includegraphics[scale=0.8, angle=90]{chp_singlecarbon/images/xbaa}
\end{textblock}
\clearpage
%=-=-=-=-=-=-=-=-=-=-=-=-=-=-=-=-=-=-=-=-=-=-=-=-=-=-=-=-=-=-=-=-=-=-=-=-=-=-=-=-=

%=[xbab]=-=-=-=-=-=-=-=-=-=-=-=-=-=-=-=-=-=-=-=-=-=-=-=-=-=-=-=-=-=-=-=-=-=-=-=-=-=-=
\begin{textblock}{20}(0,0)
\begin{figure}[htb]
\caption{$^1$H NMR of \CMPxbab\ (\ref{cmp:xbab})}
\includegraphics[scale=0.75, trim = 0mm 0mm 0mm 5mm,
clip]{chp_singlecarbon/images/nmr/xbabH}
\vspace{-100pt}
\end{figure}
\end{textblock}
\begin{textblock}{1}(2,4)
\includegraphics[scale=0.8, angle=90]{chp_singlecarbon/images/xbab}
\end{textblock}
\clearpage
%%%
\begin{textblock}{20}(0,0)
\begin{figure}[htb]
\caption{$^{13}$C NMR of  \CMPxbab\ (\ref{cmp:xbab})}
\includegraphics[scale=0.75, trim = 0mm 0mm 0mm 5mm,
clip]{chp_singlecarbon/images/nmr/xbabC}
\vspace{-100pt}
\end{figure}
\end{textblock}
\begin{textblock}{1}(2,2)
\includegraphics[scale=0.8, angle=90]{chp_singlecarbon/images/xbab}
\end{textblock}
\clearpage
%=-=-=-=-=-=-=-=-=-=-=-=-=-=-=-=-=-=-=-=-=-=-=-=-=-=-=-=-=-=-=-=-=-=-=-=-=-=-=-=-=

%=[xbac]=-=-=-=-=-=-=-=-=-=-=-=-=-=-=-=-=-=-=-=-=-=-=-=-=-=-=-=-=-=-=-=-=-=-=-=-=-=-=
\begin{textblock}{20}(0,0)
\begin{figure}[htb]
\caption{$^1$H NMR of \CMPxbac\ (\ref{cmp:xbac})}
\includegraphics[scale=0.75, trim = 0mm 0mm 0mm 5mm,
clip]{chp_singlecarbon/images/nmr/xbacH}
\vspace{-100pt}
\end{figure}
\end{textblock}
\begin{textblock}{1}(2,2)
\includegraphics[scale=0.8, angle=90]{chp_singlecarbon/images/xbac}
\end{textblock}
\clearpage
%%%
\begin{textblock}{20}(0,0)
\begin{figure}[htb]
\caption{$^{13}$C NMR of  \CMPxbac\ (\ref{cmp:xbac})}
\includegraphics[scale=0.75, trim = 0mm 0mm 0mm 5mm,
clip]{chp_singlecarbon/images/nmr/xbacC}
\vspace{-100pt}
\end{figure}
\end{textblock}
\begin{textblock}{1}(2,2)
\includegraphics[scale=0.8, angle=90]{chp_singlecarbon/images/xbac}
\end{textblock}
\clearpage
%=-=-=-=-=-=-=-=-=-=-=-=-=-=-=-=-=-=-=-=-=-=-=-=-=-=-=-=-=-=-=-=-=-=-=-=-=-=-=-=-=

%=[xbad]=-=-=-=-=-=-=-=-=-=-=-=-=-=-=-=-=-=-=-=-=-=-=-=-=-=-=-=-=-=-=-=-=-=-=-=-=-=-=
\begin{textblock}{20}(0,0)
\begin{figure}[htb]
\caption{$^1$H NMR of \CMPxbad\ (\ref{cmp:xbad})}
\includegraphics[scale=0.75, trim = 0mm 0mm 0mm 5mm,
clip]{chp_singlecarbon/images/nmr/xbadH}
\vspace{-100pt}
\end{figure}
\end{textblock}
\begin{textblock}{1}(2,2)
\includegraphics[scale=0.8, angle=90]{chp_singlecarbon/images/xbad}
\end{textblock}
\clearpage
%%%
\begin{textblock}{20}(0,0)
\begin{figure}[htb]
\caption{$^{13}$C NMR of  \CMPxbad\ (\ref{cmp:xbad})}
\includegraphics[scale=0.75, trim = 0mm 0mm 0mm 5mm,
clip]{chp_singlecarbon/images/nmr/xbadC}
\vspace{-100pt}
\end{figure}
\end{textblock}
\begin{textblock}{1}(2,2)
\includegraphics[scale=0.8, angle=90]{chp_singlecarbon/images/xbad}
\end{textblock}
\clearpage
%=-=-=-=-=-=-=-=-=-=-=-=-=-=-=-=-=-=-=-=-=-=-=-=-=-=-=-=-=-=-=-=-=-=-=-=-=-=-=-=-=

%=[xbae]=-=-=-=-=-=-=-=-=-=-=-=-=-=-=-=-=-=-=-=-=-=-=-=-=-=-=-=-=-=-=-=-=-=-=-=-=-=-=
\begin{textblock}{20}(0,0)
\begin{figure}[htb]
\caption{$^1$H NMR of \CMPxbae\ (\ref{cmp:xbae})}
\includegraphics[scale=0.75, trim = 0mm 0mm 0mm 5mm,
clip]{chp_singlecarbon/images/nmr/xbaeH}
\vspace{-100pt}
\end{figure}
\end{textblock}
\begin{textblock}{1}(2,2)
\includegraphics[scale=0.8, angle=90]{chp_singlecarbon/images/xbae}
\end{textblock}
\clearpage
%%%
\begin{textblock}{20}(0,0)
\begin{figure}[htb]
\caption{$^{13}$C NMR of  \CMPxbae\ (\ref{cmp:xbae})}
\includegraphics[scale=0.75, trim = 0mm 0mm 0mm 5mm,
clip]{chp_singlecarbon/images/nmr/xbaeC}
\vspace{-100pt}
\end{figure}
\end{textblock}
\begin{textblock}{1}(0.5,2)
\includegraphics[scale=0.8, angle=90]{chp_singlecarbon/images/xbae}
\end{textblock}
\clearpage
%=-=-=-=-=-=-=-=-=-=-=-=-=-=-=-=-=-=-=-=-=-=-=-=-=-=-=-=-=-=-=-=-=-=-=-=-=-=-=-=-=

%=[xbaf]=-=-=-=-=-=-=-=-=-=-=-=-=-=-=-=-=-=-=-=-=-=-=-=-=-=-=-=-=-=-=-=-=-=-=-=-=-=-=
\begin{textblock}{20}(0,0)
\begin{figure}[htb]
\caption{$^1$H NMR of \CMPxbaf\ (\ref{cmp:xbaf})}
\includegraphics[scale=0.75, trim = 0mm 0mm 0mm 5mm,
clip]{chp_singlecarbon/images/nmr/xbafH}
\vspace{-100pt}
\end{figure}
\end{textblock}
\begin{textblock}{1}(2,2)
\includegraphics[scale=0.8, angle=90]{chp_singlecarbon/images/xbaf}
\end{textblock}
\clearpage
%%%
\begin{textblock}{20}(0,0)
\begin{figure}[htb]
\caption{$^{13}$C NMR of  \CMPxbaf\ (\ref{cmp:xbaf})}
\includegraphics[scale=0.75, trim = 0mm 0mm 0mm 5mm,
clip]{chp_singlecarbon/images/nmr/xbafC}
\vspace{-100pt}
\end{figure}
\end{textblock}
\begin{textblock}{1}(2,2)
\includegraphics[scale=0.8, angle=90]{chp_singlecarbon/images/xbaf}
\end{textblock}
\clearpage
%=-=-=-=-=-=-=-=-=-=-=-=-=-=-=-=-=-=-=-=-=-=-=-=-=-=-=-=-=-=-=-=-=-=-=-=-=-=-=-=-=

%=[xbai]=-=-=-=-=-=-=-=-=-=-=-=-=-=-=-=-=-=-=-=-=-=-=-=-=-=-=-=-=-=-=-=-=-=-=-=-=-=-=
\begin{textblock}{20}(0,0)
\begin{figure}[htb]
\caption{$^1$H NMR of \CMPxbai\ (\ref{cmp:xbai})}
\includegraphics[scale=0.75, trim = 0mm 0mm 0mm 5mm,
clip]{chp_singlecarbon/images/nmr/xbaiH}
\vspace{-100pt}
\end{figure}
\end{textblock}
\begin{textblock}{1}(2,2)
\includegraphics[scale=0.8, angle=90]{chp_singlecarbon/images/xbai}
\end{textblock}
\clearpage
%%%
\begin{textblock}{20}(0,0)
\begin{figure}[htb]
\caption{$^{13}$C NMR of  \CMPxbai\ (\ref{cmp:xbai})}
\includegraphics[scale=0.75, trim = 0mm 0mm 0mm 5mm,
clip]{chp_singlecarbon/images/nmr/xbaiC}
\vspace{-100pt}
\end{figure}
\end{textblock}
\begin{textblock}{1}(1,2)
\includegraphics[scale=0.8, angle=90]{chp_singlecarbon/images/xbai}
\end{textblock}
\clearpage
%=-=-=-=-=-=-=-=-=-=-=-=-=-=-=-=-=-=-=-=-=-=-=-=-=-=-=-=-=-=-=-=-=-=-=-=-=-=-=-=-=

%=[xban]=-=-=-=-=-=-=-=-=-=-=-=-=-=-=-=-=-=-=-=-=-=-=-=-=-=-=-=-=-=-=-=-=-=-=-=-=-=-=
\begin{textblock}{20}(0,0)
\begin{figure}[htb]
\caption{$^1$H NMR of \CMPxban\ (\ref{cmp:xban})}
\includegraphics[scale=0.75, trim = 0mm 0mm 0mm 5mm,
clip]{chp_singlecarbon/images/nmr/xbanH}
\vspace{-100pt}
\end{figure}
\end{textblock}
\begin{textblock}{1}(2,2)
\includegraphics[scale=0.8, angle=90]{chp_singlecarbon/images/xban}
\end{textblock}
\clearpage
%%%
\begin{textblock}{20}(0,0)
\begin{figure}[htb]
\caption{$^{13}$C NMR of  \CMPxban\ (\ref{cmp:xban})}
\includegraphics[scale=0.75, trim = 0mm 0mm 0mm 5mm,
clip]{chp_singlecarbon/images/nmr/xbanC}
\vspace{-100pt}
\end{figure}
\end{textblock}
\begin{textblock}{1}(2,15)
\includegraphics[scale=0.8, angle=90]{chp_singlecarbon/images/xban}
\end{textblock}
\clearpage
%=-=-=-=-=-=-=-=-=-=-=-=-=-=-=-=-=-=-=-=-=-=-=-=-=-=-=-=-=-=-=-=-=-=-=-=-=-=-=-=-=

%=[xbaz]=-=-=-=-=-=-=-=-=-=-=-=-=-=-=-=-=-=-=-=-=-=-=-=-=-=-=-=-=-=-=-=-=-=-=-=-=-=-=
\begin{textblock}{20}(0,0)
\begin{figure}[htb]
\caption{$^1$H NMR of \CMPxbaz\ (\ref{cmp:xbaz})}
\includegraphics[scale=0.75, trim = 0mm 0mm 0mm 5mm,
clip]{chp_singlecarbon/images/nmr/xbazH}
\vspace{-100pt}
\end{figure}
\end{textblock}
\begin{textblock}{1}(2,2)
\includegraphics[scale=0.8, angle=90]{chp_singlecarbon/images/xbaz}
\end{textblock}
\clearpage
%%%
\begin{textblock}{20}(0,0)
\begin{figure}[htb]
\caption{$^{13}$C NMR of  \CMPxbaz\ (\ref{cmp:xbaz})}
\includegraphics[scale=0.75, trim = 0mm 0mm 0mm 5mm,
clip]{chp_singlecarbon/images/nmr/xbazC}
\vspace{-100pt}
\end{figure}
\end{textblock}
\begin{textblock}{1}(2,2)
\includegraphics[scale=0.8, angle=90]{chp_singlecarbon/images/xbaz}
\end{textblock}
\clearpage
%=-=-=-=-=-=-=-=-=-=-=-=-=-=-=-=-=-=-=-=-=-=-=-=-=-=-=-=-=-=-=-=-=-=-=-=-=-=-=-=-=

%=[xbbo]=-=-=-=-=-=-=-=-=-=-=-=-=-=-=-=-=-=-=-=-=-=-=-=-=-=-=-=-=-=-=-=-=-=-=-=-=-=-=
\begin{textblock}{20}(0,0)
\begin{figure}[htb]
\caption{$^1$H NMR of \CMPxbbo\ (\ref{cmp:xbbo})}
\includegraphics[scale=0.75, trim = 0mm 0mm 0mm 5mm,
clip]{chp_singlecarbon/images/nmr/xbboH}
\vspace{-100pt}
\end{figure}
\end{textblock}
\begin{textblock}{1}(2,2)
\includegraphics[scale=0.8, angle=90]{chp_singlecarbon/images/xbbo}
\end{textblock}
\clearpage
%%%
\begin{textblock}{20}(0,0)
\begin{figure}[htb]
\caption{$^{13}$C NMR of  \CMPxbbo\ (\ref{cmp:xbbo})}
\includegraphics[scale=0.75, trim = 0mm 0mm 0mm 5mm,
clip]{chp_singlecarbon/images/nmr/xbboC}
\vspace{-100pt}
\end{figure}
\end{textblock}
\begin{textblock}{1}(2,2)
\includegraphics[scale=0.8, angle=90]{chp_singlecarbon/images/xbbo}
\end{textblock}
\clearpage
%=-=-=-=-=-=-=-=-=-=-=-=-=-=-=-=-=-=-=-=-=-=-=-=-=-=-=-=-=-=-=-=-=-=-=-=-=-=-=-=-=

%=[xbaj]=-=-=-=-=-=-=-=-=-=-=-=-=-=-=-=-=-=-=-=-=-=-=-=-=-=-=-=-=-=-=-=-=-=-=-=-=-=-=
\begin{textblock}{20}(0,0)
\begin{figure}[htb]
\caption{$^1$H NMR of \CMPxbaj\ (\ref{cmp:xbaj})}
\includegraphics[scale=0.75, trim = 0mm 0mm 0mm 5mm,
clip]{chp_singlecarbon/images/nmr/xbajH}
\vspace{-100pt}
\end{figure}
\end{textblock}
\begin{textblock}{1}(2,2)
\includegraphics[scale=0.8, angle=90]{chp_singlecarbon/images/xbaj}
\end{textblock}
\clearpage
%%%
\begin{textblock}{20}(0,0)
\begin{figure}[htb]
\caption{$^{13}$C NMR of  \CMPxbaj\ (\ref{cmp:xbaj})}
\includegraphics[scale=0.75, trim = 0mm 0mm 0mm 5mm,
clip]{chp_singlecarbon/images/nmr/xbajC}
\vspace{-100pt}
\end{figure}
\end{textblock}
\begin{textblock}{1}(2,2)
\includegraphics[scale=0.8, angle=90]{chp_singlecarbon/images/xbaj}
\end{textblock}
\clearpage
%=-=-=-=-=-=-=-=-=-=-=-=-=-=-=-=-=-=-=-=-=-=-=-=-=-=-=-=-=-=-=-=-=-=-=-=-=-=-=-=-=

%=[xbak]=-=-=-=-=-=-=-=-=-=-=-=-=-=-=-=-=-=-=-=-=-=-=-=-=-=-=-=-=-=-=-=-=-=-=-=-=-=-=
\begin{textblock}{20}(0,0)
\begin{figure}[htb]
\caption{$^1$H NMR of \CMPxbak\ (\ref{cmp:xbak})}
\includegraphics[scale=0.75, trim = 0mm 0mm 0mm 5mm,
clip]{chp_singlecarbon/images/nmr/xbakH}
\vspace{-100pt}
\end{figure}
\end{textblock}
\begin{textblock}{1}(2,2)
\includegraphics[scale=0.8, angle=90]{chp_singlecarbon/images/xbak}
\end{textblock}
\clearpage
%%%
\begin{textblock}{20}(0,0)
\begin{figure}[htb]
\caption{$^{13}$C NMR of  \CMPxbak\ (\ref{cmp:xbak})}
\includegraphics[scale=0.75, trim = 0mm 0mm 0mm 5mm,
clip]{chp_singlecarbon/images/nmr/xbakC}
\vspace{-100pt}
\end{figure}
\end{textblock}
\begin{textblock}{1}(2,2)
\includegraphics[scale=0.8, angle=90]{chp_singlecarbon/images/xbak}
\end{textblock}
\clearpage
%=-=-=-=-=-=-=-=-=-=-=-=-=-=-=-=-=-=-=-=-=-=-=-=-=-=-=-=-=-=-=-=-=-=-=-=-=-=-=-=-=

%=[xbal]=-=-=-=-=-=-=-=-=-=-=-=-=-=-=-=-=-=-=-=-=-=-=-=-=-=-=-=-=-=-=-=-=-=-=-=-=-=-=
\begin{textblock}{20}(0,0)
\begin{figure}[htb]
\caption{$^1$H NMR of \CMPxbal\ (\ref{cmp:xbal})}
\includegraphics[scale=0.75, trim = 0mm 0mm 0mm 5mm,
clip]{chp_singlecarbon/images/nmr/xbalH}
\vspace{-100pt}
\end{figure}
\end{textblock}
\begin{textblock}{1}(2,2)
\includegraphics[scale=0.8, angle=90]{chp_singlecarbon/images/xbal}
\end{textblock}
\clearpage
%%%
\begin{textblock}{20}(0,0)
\begin{figure}[htb]
\caption{$^{13}$C NMR of  \CMPxbal\ (\ref{cmp:xbal})}
\includegraphics[scale=0.75, trim = 0mm 0mm 0mm 5mm,
clip]{chp_singlecarbon/images/nmr/xbalC}
\vspace{-100pt}
\end{figure}
\end{textblock}
\begin{textblock}{1}(2,2)
\includegraphics[scale=0.8, angle=90]{chp_singlecarbon/images/xbal}
\end{textblock}
\clearpage
%=-=-=-=-=-=-=-=-=-=-=-=-=-=-=-=-=-=-=-=-=-=-=-=-=-=-=-=-=-=-=-=-=-=-=-=-=-=-=-=-=

%=[xbam]=-=-=-=-=-=-=-=-=-=-=-=-=-=-=-=-=-=-=-=-=-=-=-=-=-=-=-=-=-=-=-=-=-=-=-=-=-=-=
\begin{textblock}{20}(0,0)
\begin{figure}[htb]
\caption{$^1$H NMR of \CMPxbam\ (\ref{cmp:xbam})}
\includegraphics[scale=0.75, trim = 0mm 0mm 0mm 5mm,
clip]{chp_singlecarbon/images/nmr/xbamH}
\vspace{-100pt}
\end{figure}
\end{textblock}
\begin{textblock}{1}(2,2)
\includegraphics[scale=0.8, angle=90]{chp_singlecarbon/images/xbam}
\end{textblock}
\clearpage
%%%
\begin{textblock}{20}(0,0)
\begin{figure}[htb]
\caption{$^{13}$C NMR of  \CMPxbam\ (\ref{cmp:xbam})}
\includegraphics[scale=0.75, trim = 0mm 0mm 0mm 5mm,
clip]{chp_singlecarbon/images/nmr/xbamC}
\vspace{-100pt}
\end{figure}
\end{textblock}
\begin{textblock}{1}(2,2)
\includegraphics[scale=0.8, angle=90]{chp_singlecarbon/images/xbam}
\end{textblock}
\clearpage
%=-=-=-=-=-=-=-=-=-=-=-=-=-=-=-=-=-=-=-=-=-=-=-=-=-=-=-=-=-=-=-=-=-=-=-=-=-=-=-=-=

%=[xbao]=-=-=-=-=-=-=-=-=-=-=-=-=-=-=-=-=-=-=-=-=-=-=-=-=-=-=-=-=-=-=-=-=-=-=-=-=-=-=
\begin{textblock}{20}(0,0)
\begin{figure}[htb]
\caption{$^1$H NMR of \CMPxbao\ (\ref{cmp:xbao})}
\includegraphics[scale=0.75, trim = 0mm 0mm 0mm 5mm,
clip]{chp_singlecarbon/images/nmr/xbaoH}
\vspace{-100pt}
\end{figure}
\end{textblock}
\begin{textblock}{1}(2,2)
\includegraphics[scale=0.8, angle=90]{chp_singlecarbon/images/xbao}
\end{textblock}
\clearpage
%%%
\begin{textblock}{20}(0,0)
\begin{figure}[htb]
\caption{$^{13}$C NMR of  \CMPxbao\ (\ref{cmp:xbao})}
\includegraphics[scale=0.75, trim = 0mm 0mm 0mm 5mm,
clip]{chp_singlecarbon/images/nmr/xbaoC}
\vspace{-100pt}
\end{figure}
\end{textblock}
\begin{textblock}{1}(2,2)
\includegraphics[scale=0.8, angle=90]{chp_singlecarbon/images/xbao}
\end{textblock}
\clearpage
%%%
\begin{textblock}{20}(0,0)
\begin{figure}[htb]
\caption{1D TOCSY NMR of  \CMPxbao\ (\ref{cmp:xbao})}
\includegraphics[scale=0.35, trim = 20mm 15mm 10mm 15mm,
clip, angle=270]{chp_singlecarbon/images/nmr/xbaoH} \\
\includegraphics[scale=0.35, trim = 20mm 0mm 10mm 15mm,
clip, angle=270]{chp_singlecarbon/images/nmr/xbaoTvinyl} \\
\includegraphics[scale=0.35, trim = 20mm 0mm 10mm 15mm,
clip, angle=270]{chp_singlecarbon/images/nmr/xbaoTcarbinol}
\vspace{-100pt}
\end{figure}
\end{textblock}
\begin{textblock}{1}(14,3)
\includegraphics[scale=0.8]{chp_singlecarbon/images/xbao}
\end{textblock}
\begin{textblock}{9}(12.5,13)
\centering \textsf{\small \textit{Irradiation of vinyl protons}\\5H in spin system}
\end{textblock}
\begin{textblock}{9}(12.5,22)
\centering \textsf{\small \textit{Irradiation of carbinol proton}\\4H in spin system}
\end{textblock}
\clearpage
%=-=-=-=-=-=-=-=-=-=-=-=-=-=-=-=-=-=-=-=-=-=-=-=-=-=-=-=-=-=-=-=-=-=-=-=-=-=-=-=-=

%=[xbap]=-=-=-=-=-=-=-=-=-=-=-=-=-=-=-=-=-=-=-=-=-=-=-=-=-=-=-=-=-=-=-=-=-=-=-=-=-=-=
\begin{textblock}{20}(0,0)
\begin{figure}[htb]
\caption{$^1$H NMR of \CMPxbap\ (\ref{cmp:xbap})}
\includegraphics[scale=0.75, trim = 0mm 0mm 0mm 5mm,
clip]{chp_singlecarbon/images/nmr/xbapH}
\vspace{-100pt}
\end{figure}
\end{textblock}
\begin{textblock}{1}(2,14)
\includegraphics[scale=0.8, angle=90]{chp_singlecarbon/images/xbap}
\end{textblock}
\clearpage
%%%
\begin{textblock}{20}(0,0)
\begin{figure}[htb]
\caption{$^{13}$C NMR of  \CMPxbap\ (\ref{cmp:xbap})}
\includegraphics[scale=0.75, trim = 0mm 0mm 0mm 5mm,
clip]{chp_singlecarbon/images/nmr/xbapC}
\vspace{-100pt}
\end{figure}
\end{textblock}
\begin{textblock}{1}(2,14)
\includegraphics[scale=0.8, angle=90]{chp_singlecarbon/images/xbap}
\end{textblock}
\clearpage
%=-=-=-=-=-=-=-=-=-=-=-=-=-=-=-=-=-=-=-=-=-=-=-=-=-=-=-=-=-=-=-=-=-=-=-=-=-=-=-=-=

%=[xbaq]=-=-=-=-=-=-=-=-=-=-=-=-=-=-=-=-=-=-=-=-=-=-=-=-=-=-=-=-=-=-=-=-=-=-=-=-=-=-=
\begin{textblock}{20}(0,0)
\begin{figure}[htb]
\caption{$^1$H NMR of \CMPxbaq\ (\ref{cmp:xbaq})}
\includegraphics[scale=0.75, trim = 0mm 0mm 0mm 5mm,
clip]{chp_singlecarbon/images/nmr/xbaqH}
\vspace{-100pt}
\end{figure}
\end{textblock}
\begin{textblock}{1}(2,14)
\includegraphics[scale=0.8, angle=90]{chp_singlecarbon/images/xbaq}
\end{textblock}
\clearpage
%%%
\begin{textblock}{20}(0,0)
\begin{figure}[htb]
\caption{$^{13}$C NMR of  \CMPxbaq\ (\ref{cmp:xbaq})}
\includegraphics[scale=0.75, trim = 0mm 0mm 0mm 5mm,
clip]{chp_singlecarbon/images/nmr/xbaqC}
\vspace{-100pt}
\end{figure}
\end{textblock}
\begin{textblock}{1}(2,14)
\includegraphics[scale=0.8, angle=90]{chp_singlecarbon/images/xbaq}
\end{textblock}
\clearpage
%=-=-=-=-=-=-=-=-=-=-=-=-=-=-=-=-=-=-=-=-=-=-=-=-=-=-=-=-=-=-=-=-=-=-=-=-=-=-=-=-=

%=[xbar]=-=-=-=-=-=-=-=-=-=-=-=-=-=-=-=-=-=-=-=-=-=-=-=-=-=-=-=-=-=-=-=-=-=-=-=-=-=-=
\begin{textblock}{20}(0,0)
\begin{figure}[htb]
\caption{$^1$H NMR of \CMPxbar\ (\ref{cmp:xbar})}
\includegraphics[scale=0.75, trim = 0mm 0mm 0mm 5mm,
clip]{chp_singlecarbon/images/nmr/xbarH}
\vspace{-100pt}
\end{figure}
\end{textblock}
\begin{textblock}{1}(2,2)
\includegraphics[scale=0.8, angle=90]{chp_singlecarbon/images/xbar}
\end{textblock}
\clearpage
%%%
\begin{textblock}{20}(0,0)
\begin{figure}[htb]
\caption{$^{13}$C NMR of  \CMPxbar\ (\ref{cmp:xbar})}
\includegraphics[scale=0.75, trim = 0mm 0mm 0mm 5mm,
clip]{chp_singlecarbon/images/nmr/xbarC}
\vspace{-100pt}
\end{figure}
\end{textblock}
\begin{textblock}{1}(2,2)
\includegraphics[scale=0.8, angle=90]{chp_singlecarbon/images/xbar}
\end{textblock}
\clearpage
%=-=-=-=-=-=-=-=-=-=-=-=-=-=-=-=-=-=-=-=-=-=-=-=-=-=-=-=-=-=-=-=-=-=-=-=-=-=-=-=-=

%=[xbas]=-=-=-=-=-=-=-=-=-=-=-=-=-=-=-=-=-=-=-=-=-=-=-=-=-=-=-=-=-=-=-=-=-=-=-=-=-=-=
\begin{textblock}{20}(0,0)
\begin{figure}[htb]
\caption{$^1$H NMR of \CMPxbas\ (\ref{cmp:xbas})}
\includegraphics[scale=0.75, trim = 0mm 0mm 0mm 5mm,
clip]{chp_singlecarbon/images/nmr/xbasH}
\vspace{-100pt}
\end{figure}
\end{textblock}
\begin{textblock}{1}(2,2)
\includegraphics[scale=0.8, angle=90]{chp_singlecarbon/images/xbas}
\end{textblock}
\clearpage
%%%
\begin{textblock}{20}(0,0)
\begin{figure}[htb]
\caption{$^{13}$C NMR of  \CMPxbas\ (\ref{cmp:xbas})}
\includegraphics[scale=0.75, trim = 0mm 0mm 0mm 5mm,
clip]{chp_singlecarbon/images/nmr/xbasC}
\vspace{-100pt}
\end{figure}
\end{textblock}
\begin{textblock}{1}(1,14)
\includegraphics[scale=0.8, angle=90]{chp_singlecarbon/images/xbas}
\end{textblock}
\clearpage
%=-=-=-=-=-=-=-=-=-=-=-=-=-=-=-=-=-=-=-=-=-=-=-=-=-=-=-=-=-=-=-=-=-=-=-=-=-=-=-=-=

%=[xbat]=-=-=-=-=-=-=-=-=-=-=-=-=-=-=-=-=-=-=-=-=-=-=-=-=-=-=-=-=-=-=-=-=-=-=-=-=-=-=
\begin{textblock}{20}(0,0)
\begin{figure}[htb]
\caption{$^1$H NMR of \CMPxbat\ (\ref{cmp:xbat})}
\includegraphics[scale=0.75, trim = 0mm 0mm 0mm 5mm,
clip]{chp_singlecarbon/images/nmr/xbatH}
\vspace{-100pt}
\end{figure}
\end{textblock}
\begin{textblock}{1}(2,2)
\includegraphics[scale=0.8, angle=90]{chp_singlecarbon/images/xbat}
\end{textblock}
\clearpage
%%%
\begin{textblock}{20}(0,0)
\begin{figure}[htb]
\caption{$^{13}$C NMR of  \CMPxbat\ (\ref{cmp:xbat})}
\includegraphics[scale=0.75, trim = 0mm 0mm 0mm 5mm,
clip]{chp_singlecarbon/images/nmr/xbatC}
\vspace{-100pt}
\end{figure}
\end{textblock}
\begin{textblock}{1}(2,2)
\includegraphics[scale=0.8, angle=90]{chp_singlecarbon/images/xbat}
\end{textblock}
\clearpage
%=-=-=-=-=-=-=-=-=-=-=-=-=-=-=-=-=-=-=-=-=-=-=-=-=-=-=-=-=-=-=-=-=-=-=-=-=-=-=-=-=

%=[xbau]=-=-=-=-=-=-=-=-=-=-=-=-=-=-=-=-=-=-=-=-=-=-=-=-=-=-=-=-=-=-=-=-=-=-=-=-=-=-=
\begin{textblock}{20}(0,0)
\begin{figure}[htb]
\caption{$^1$H NMR of \CMPxbau\ (\ref{cmp:xbau})}
\includegraphics[scale=0.75, trim = 0mm 0mm 0mm 5mm,
clip]{chp_singlecarbon/images/nmr/xbauH}
\vspace{-100pt}
\end{figure}
\end{textblock}
\begin{textblock}{1}(2,2)
\includegraphics[scale=0.8, angle=90]{chp_singlecarbon/images/xbau}
\end{textblock}
\clearpage
%%%
\begin{textblock}{20}(0,0)
\begin{figure}[htb]
\caption{$^{13}$C NMR of  \CMPxbau\ (\ref{cmp:xbau})}
\includegraphics[scale=0.75, trim = 0mm 0mm 0mm 5mm,
clip]{chp_singlecarbon/images/nmr/xbauC}
\vspace{-100pt}
\end{figure}
\end{textblock}
\begin{textblock}{1}(2,2)
\includegraphics[scale=0.8, angle=90]{chp_singlecarbon/images/xbau}
\end{textblock}
\clearpage
%=-=-=-=-=-=-=-=-=-=-=-=-=-=-=-=-=-=-=-=-=-=-=-=-=-=-=-=-=-=-=-=-=-=-=-=-=-=-=-=-=

%=[xbav]=-=-=-=-=-=-=-=-=-=-=-=-=-=-=-=-=-=-=-=-=-=-=-=-=-=-=-=-=-=-=-=-=-=-=-=-=-=-=
\begin{textblock}{20}(0,0)
\begin{figure}[htb]
\caption{$^1$H NMR of \CMPxbav\ (\ref{cmp:xbav})}
\includegraphics[scale=0.75, trim = 0mm 0mm 0mm 5mm,
clip]{chp_singlecarbon/images/nmr/xbavH}
\vspace{-100pt}
\end{figure}
\end{textblock}
\begin{textblock}{1}(2,2)
\includegraphics[scale=0.8, angle=90]{chp_singlecarbon/images/xbav}
\end{textblock}
\clearpage
%%%
\begin{textblock}{20}(0,0)
\begin{figure}[htb]
\caption{$^{13}$C NMR of  \CMPxbav\ (\ref{cmp:xbav})}
\includegraphics[scale=0.75, trim = 0mm 0mm 0mm 5mm,
clip]{chp_singlecarbon/images/nmr/xbavC}
\vspace{-100pt}
\end{figure}
\end{textblock}
\begin{textblock}{1}(2,2)
\includegraphics[scale=0.8, angle=90]{chp_singlecarbon/images/xbav}
\end{textblock}
\clearpage
%=-=-=-=-=-=-=-=-=-=-=-=-=-=-=-=-=-=-=-=-=-=-=-=-=-=-=-=-=-=-=-=-=-=-=-=-=-=-=-=-=

%=[xbaw]=-=-=-=-=-=-=-=-=-=-=-=-=-=-=-=-=-=-=-=-=-=-=-=-=-=-=-=-=-=-=-=-=-=-=-=-=-=-=
\begin{textblock}{20}(0,0)
\begin{figure}[htb]
\caption{$^1$H NMR of \CMPxbaw\ (\ref{cmp:xbaw})}
\includegraphics[scale=0.75, trim = 0mm 0mm 0mm 5mm,
clip]{chp_singlecarbon/images/nmr/xbawH}
\vspace{-100pt}
\end{figure}
\end{textblock}
\begin{textblock}{1}(2,2)
\includegraphics[scale=0.8, angle=90]{chp_singlecarbon/images/xbaw}
\end{textblock}
\clearpage
%%%
\begin{textblock}{20}(0,0)
\begin{figure}[htb]
\caption{$^{13}$C NMR of  \CMPxbaw\ (\ref{cmp:xbaw})}
\includegraphics[scale=0.75, trim = 0mm 0mm 0mm 5mm,
clip]{chp_singlecarbon/images/nmr/xbawC}
\vspace{-100pt}
\end{figure}
\end{textblock}
\begin{textblock}{1}(2,2)
\includegraphics[scale=0.8, angle=90]{chp_singlecarbon/images/xbaw}
\end{textblock}
\clearpage
%=-=-=-=-=-=-=-=-=-=-=-=-=-=-=-=-=-=-=-=-=-=-=-=-=-=-=-=-=-=-=-=-=-=-=-=-=-=-=-=-=

%=[xbax]=-=-=-=-=-=-=-=-=-=-=-=-=-=-=-=-=-=-=-=-=-=-=-=-=-=-=-=-=-=-=-=-=-=-=-=-=-=-=
\begin{textblock}{20}(0,0)
\begin{figure}[htb]
\caption{$^1$H NMR of \CMPxbax\ (\ref{cmp:xbax})}
\includegraphics[scale=0.75, trim = 0mm 0mm 0mm 5mm,
clip]{chp_singlecarbon/images/nmr/xbaxH}
\vspace{-100pt}
\end{figure}
\end{textblock}
\begin{textblock}{1}(2,2)
\includegraphics[scale=0.8, angle=90]{chp_singlecarbon/images/xbax}
\end{textblock}
\clearpage
%%%
\begin{textblock}{20}(0,0)
\begin{figure}[htb]
\caption{$^{13}$C NMR of  \CMPxbax\ (\ref{cmp:xbax})}
\includegraphics[scale=0.75, trim = 0mm 0mm 0mm 5mm,
clip]{chp_singlecarbon/images/nmr/xbaxC}
\vspace{-100pt}
\end{figure}
\end{textblock}
\begin{textblock}{1}(2,2)
\includegraphics[scale=0.8, angle=90]{chp_singlecarbon/images/xbax}
\end{textblock}
\clearpage
%=-=-=-=-=-=-=-=-=-=-=-=-=-=-=-=-=-=-=-=-=-=-=-=-=-=-=-=-=-=-=-=-=-=-=-=-=-=-=-=-=

%=[xbay]=-=-=-=-=-=-=-=-=-=-=-=-=-=-=-=-=-=-=-=-=-=-=-=-=-=-=-=-=-=-=-=-=-=-=-=-=-=-=
\begin{textblock}{20}(0,0)
\begin{figure}[htb]
\caption{$^1$H NMR of \CMPxbay\ (\ref{cmp:xbay})}
\includegraphics[scale=0.75, trim = 0mm 0mm 0mm 5mm,
clip]{chp_singlecarbon/images/nmr/xbayH}
\vspace{-100pt}
\end{figure}
\end{textblock}
\begin{textblock}{1}(2,2)
\includegraphics[scale=0.8, angle=90]{chp_singlecarbon/images/xbay}
\end{textblock}
\clearpage
%%%
\begin{textblock}{20}(0,0)
\begin{figure}[htb]
\caption{$^{13}$C NMR of  \CMPxbay\ (\ref{cmp:xbay})}
\includegraphics[scale=0.75, trim = 0mm 0mm 0mm 5mm,
clip]{chp_singlecarbon/images/nmr/xbayC}
\vspace{-100pt}
\end{figure}
\end{textblock}
\begin{textblock}{1}(2,2)
\includegraphics[scale=0.8, angle=90]{chp_singlecarbon/images/xbay}
\end{textblock}
\clearpage
%=-=-=-=-=-=-=-=-=-=-=-=-=-=-=-=-=-=-=-=-=-=-=-=-=-=-=-=-=-=-=-=-=-=-=-=-=-=-=-=-=

%xbba-xbbj 10

%=[xbba]=-=-=-=-=-=-=-=-=-=-=-=-=-=-=-=-=-=-=-=-=-=-=-=-=-=-=-=-=-=-=-=-=-=-=-=-=-=-=
\begin{textblock}{20}(0,0)
\begin{figure}[htb]
\caption{$^1$H NMR of \CMPxbba\ (\ref{cmp:xbba})}
\includegraphics[scale=0.75, trim = 0mm 0mm 0mm 5mm,
clip]{chp_singlecarbon/images/nmr/xbbaH}
\vspace{-100pt}
\end{figure}
\end{textblock}
\begin{textblock}{1}(2,2)
\includegraphics[scale=0.8, angle=90]{chp_singlecarbon/images/xbba}
\end{textblock}
\clearpage
%%%
\begin{textblock}{20}(0,0)
\begin{figure}[htb]
\caption{$^{13}$C NMR of  \CMPxbba\ (\ref{cmp:xbba})}
\includegraphics[scale=0.75, trim = 0mm 0mm 0mm 5mm,
clip]{chp_singlecarbon/images/nmr/xbbaC}
\vspace{-100pt}
\end{figure}
\end{textblock}
\begin{textblock}{1}(0,2)
\includegraphics[scale=0.8, angle=90]{chp_singlecarbon/images/xbba}
\end{textblock}
\clearpage
%=-=-=-=-=-=-=-=-=-=-=-=-=-=-=-=-=-=-=-=-=-=-=-=-=-=-=-=-=-=-=-=-=-=-=-=-=-=-=-=-=

%=[xbbb]=-=-=-=-=-=-=-=-=-=-=-=-=-=-=-=-=-=-=-=-=-=-=-=-=-=-=-=-=-=-=-=-=-=-=-=-=-=-=
\begin{textblock}{20}(0,0)
\begin{figure}[htb]
\caption{$^1$H NMR of \CMPxbbb\ (\ref{cmp:xbbb})}
\includegraphics[scale=0.75, trim = 0mm 0mm 0mm 5mm,
clip]{chp_singlecarbon/images/nmr/xbbbH}
\vspace{-100pt}
\end{figure}
\end{textblock}
\begin{textblock}{1}(2,6)
\includegraphics[scale=0.8, angle=90]{chp_singlecarbon/images/xbbb}
\end{textblock}
\clearpage
%%%
\begin{textblock}{20}(0,0)
\begin{figure}[htb]
\caption{$^{13}$C NMR of  \CMPxbbb\ (\ref{cmp:xbbb})}
\includegraphics[scale=0.75, trim = 0mm 0mm 0mm 5mm,
clip]{chp_singlecarbon/images/nmr/xbbbC}
\vspace{-100pt}
\end{figure}
\end{textblock}
\begin{textblock}{1}(2,7)
\includegraphics[scale=0.8, angle=90]{chp_singlecarbon/images/xbbb}
\end{textblock}
\clearpage
%=-=-=-=-=-=-=-=-=-=-=-=-=-=-=-=-=-=-=-=-=-=-=-=-=-=-=-=-=-=-=-=-=-=-=-=-=-=-=-=-=

%=[xbbc]=-=-=-=-=-=-=-=-=-=-=-=-=-=-=-=-=-=-=-=-=-=-=-=-=-=-=-=-=-=-=-=-=-=-=-=-=-=-=
\begin{textblock}{20}(0,0)
\begin{figure}[htb]
\caption{$^1$H NMR of \CMPxbbc\ (\ref{cmp:xbbc})}
\includegraphics[scale=0.75, trim = 0mm 0mm 0mm 5mm,
clip]{chp_singlecarbon/images/nmr/xbbcH}
\vspace{-100pt}
\end{figure}
\end{textblock}
\begin{textblock}{1}(2,6)
\includegraphics[scale=0.8, angle=90]{chp_singlecarbon/images/xbbc}
\end{textblock}
\clearpage
%%%
\begin{textblock}{20}(0,0)
\begin{figure}[htb]
\caption{$^{13}$C NMR of  \CMPxbbc\ (\ref{cmp:xbbc})}
\includegraphics[scale=0.75, trim = 0mm 0mm 0mm 5mm,
clip]{chp_singlecarbon/images/nmr/xbbcC}
\vspace{-100pt}
\end{figure}
\end{textblock}
\begin{textblock}{1}(2,7)
\includegraphics[scale=0.8, angle=90]{chp_singlecarbon/images/xbbc}
\end{textblock}
\clearpage
%=-=-=-=-=-=-=-=-=-=-=-=-=-=-=-=-=-=-=-=-=-=-=-=-=-=-=-=-=-=-=-=-=-=-=-=-=-=-=-=-=

%=[xbbd]=-=-=-=-=-=-=-=-=-=-=-=-=-=-=-=-=-=-=-=-=-=-=-=-=-=-=-=-=-=-=-=-=-=-=-=-=-=-=
\begin{textblock}{20}(0,0)
\begin{figure}[htb]
\caption{$^1$H NMR of \CMPxbbd\ (\ref{cmp:xbbd})}
\includegraphics[scale=0.75, trim = 0mm 0mm 0mm 5mm,
clip]{chp_singlecarbon/images/nmr/xbbdH}
\vspace{-100pt}
\end{figure}
\end{textblock}
\begin{textblock}{1}(2,2)
\includegraphics[scale=0.8, angle=90]{chp_singlecarbon/images/xbbd}
\end{textblock}
\clearpage
%%%
\begin{textblock}{20}(0,0)
\begin{figure}[htb]
\caption{$^{13}$C NMR of  \CMPxbbd\ (\ref{cmp:xbbd})}
\includegraphics[scale=0.75, trim = 0mm 0mm 0mm 5mm,
clip]{chp_singlecarbon/images/nmr/xbbdC}
\vspace{-100pt}
\end{figure}
\end{textblock}
\begin{textblock}{1}(2,2)
\includegraphics[scale=0.8, angle=90]{chp_singlecarbon/images/xbbd}
\end{textblock}
\clearpage
%=-=-=-=-=-=-=-=-=-=-=-=-=-=-=-=-=-=-=-=-=-=-=-=-=-=-=-=-=-=-=-=-=-=-=-=-=-=-=-=-=

%=[xbbe]=-=-=-=-=-=-=-=-=-=-=-=-=-=-=-=-=-=-=-=-=-=-=-=-=-=-=-=-=-=-=-=-=-=-=-=-=-=-=
\begin{textblock}{20}(0,0)
\begin{figure}[htb]
\caption{$^1$H NMR of \CMPxbbe\ (\ref{cmp:xbbe})}
\includegraphics[scale=0.75, trim = 0mm 0mm 0mm 5mm,
clip]{chp_singlecarbon/images/nmr/xbbeH}
\vspace{-100pt}
\end{figure}
\end{textblock}
\begin{textblock}{1}(2,2)
\includegraphics[scale=0.8, angle=90]{chp_singlecarbon/images/xbbe}
\end{textblock}
\clearpage
%%%
\begin{textblock}{20}(0,0)
\begin{figure}[htb]
\caption{$^{13}$C NMR of  \CMPxbbe\ (\ref{cmp:xbbe})}
\includegraphics[scale=0.75, trim = 0mm 0mm 0mm 5mm,
clip]{chp_singlecarbon/images/nmr/xbbeC}
\vspace{-100pt}
\end{figure}
\end{textblock}
\begin{textblock}{1}(1,2)
\includegraphics[scale=0.8, angle=90]{chp_singlecarbon/images/xbbe}
\end{textblock}
\clearpage
%=-=-=-=-=-=-=-=-=-=-=-=-=-=-=-=-=-=-=-=-=-=-=-=-=-=-=-=-=-=-=-=-=-=-=-=-=-=-=-=-=

%=[xbbf]=-=-=-=-=-=-=-=-=-=-=-=-=-=-=-=-=-=-=-=-=-=-=-=-=-=-=-=-=-=-=-=-=-=-=-=-=-=-=
\begin{textblock}{20}(0,0)
\begin{figure}[htb]
\caption{$^1$H NMR of \CMPxbbf\ (\ref{cmp:xbbf})}
\includegraphics[scale=0.75, trim = 0mm 0mm 0mm 5mm,
clip]{chp_singlecarbon/images/nmr/xbbfH}
\vspace{-100pt}
\end{figure}
\end{textblock}
\begin{textblock}{1}(2,2)
\includegraphics[scale=0.8, angle=90]{chp_singlecarbon/images/xbbf}
\end{textblock}
\clearpage
%%%
\begin{textblock}{20}(0,0)
\begin{figure}[htb]
\caption{$^{13}$C NMR of  \CMPxbbf\ (\ref{cmp:xbbf})}
\includegraphics[scale=0.75, trim = 0mm 0mm 0mm 5mm,
clip]{chp_singlecarbon/images/nmr/xbbfC}
\vspace{-100pt}
\end{figure}
\end{textblock}
\begin{textblock}{1}(2,2)
\includegraphics[scale=0.8, angle=90]{chp_singlecarbon/images/xbbf}
\end{textblock}
\clearpage
%=-=-=-=-=-=-=-=-=-=-=-=-=-=-=-=-=-=-=-=-=-=-=-=-=-=-=-=-=-=-=-=-=-=-=-=-=-=-=-=-=

%=[xbbg]=-=-=-=-=-=-=-=-=-=-=-=-=-=-=-=-=-=-=-=-=-=-=-=-=-=-=-=-=-=-=-=-=-=-=-=-=-=-=
\begin{textblock}{20}(0,0)
\begin{figure}[htb]
\caption{$^1$H NMR of \CMPxbbg\ (\ref{cmp:xbbg})}
\includegraphics[scale=0.75, trim = 0mm 0mm 0mm 5mm,
clip]{chp_singlecarbon/images/nmr/xbbgH}
\vspace{-100pt}
\end{figure}
\end{textblock}
\begin{textblock}{1}(2,2)
\includegraphics[scale=0.8, angle=90]{chp_singlecarbon/images/xbbg}
\end{textblock}
\clearpage
%%%
\begin{textblock}{20}(0,0)
\begin{figure}[htb]
\caption{$^{13}$C NMR of  \CMPxbbg\ (\ref{cmp:xbbg})}
\includegraphics[scale=0.75, trim = 0mm 0mm 0mm 5mm,
clip]{chp_singlecarbon/images/nmr/xbbgC}
\vspace{-100pt}
\end{figure}
\end{textblock}
\begin{textblock}{1}(2,2)
\includegraphics[scale=0.8, angle=90]{chp_singlecarbon/images/xbbg}
\end{textblock}
\clearpage
%=-=-=-=-=-=-=-=-=-=-=-=-=-=-=-=-=-=-=-=-=-=-=-=-=-=-=-=-=-=-=-=-=-=-=-=-=-=-=-=-=

%=[xbbh]=-=-=-=-=-=-=-=-=-=-=-=-=-=-=-=-=-=-=-=-=-=-=-=-=-=-=-=-=-=-=-=-=-=-=-=-=-=-=
\begin{textblock}{20}(0,0)
\begin{figure}[htb]
\caption{$^1$H NMR of \CMPxbbh\ (\ref{cmp:xbbh})}
\includegraphics[scale=0.75, trim = 0mm 0mm 0mm 5mm,
clip]{chp_singlecarbon/images/nmr/xbbhH}
\vspace{-100pt}
\end{figure}
\end{textblock}
\begin{textblock}{1}(2,2)
\includegraphics[scale=0.8, angle=90]{chp_singlecarbon/images/xbbh}
\end{textblock}
\clearpage
%%%
\begin{textblock}{20}(0,0)
\begin{figure}[htb]
\caption{$^{13}$C NMR of  \CMPxbbh\ (\ref{cmp:xbbh})}
\includegraphics[scale=0.75, trim = 0mm 0mm 0mm 5mm,
clip]{chp_singlecarbon/images/nmr/xbbhC}
\vspace{-100pt}
\end{figure}
\end{textblock}
\begin{textblock}{1}(2,2)
\includegraphics[scale=0.8, angle=90]{chp_singlecarbon/images/xbbh}
\end{textblock}
\clearpage
%=-=-=-=-=-=-=-=-=-=-=-=-=-=-=-=-=-=-=-=-=-=-=-=-=-=-=-=-=-=-=-=-=-=-=-=-=-=-=-=-=

%=[xbbi]=-=-=-=-=-=-=-=-=-=-=-=-=-=-=-=-=-=-=-=-=-=-=-=-=-=-=-=-=-=-=-=-=-=-=-=-=-=-=
\begin{textblock}{20}(0,0)
\begin{figure}[htb]
\caption{$^1$H NMR of \CMPxbbi\ (\ref{cmp:xbbi})}
\includegraphics[scale=0.75, trim = 0mm 0mm 0mm 5mm,
clip]{chp_singlecarbon/images/nmr/xbbiH}
\vspace{-100pt}
\end{figure}
\end{textblock}
\begin{textblock}{1}(2,2)
\includegraphics[scale=0.8, angle=90]{chp_singlecarbon/images/xbbi}
\end{textblock}
\clearpage
%%%
\begin{textblock}{20}(0,0)
\begin{figure}[htb]
\caption{$^{13}$C NMR of  \CMPxbbi\ (\ref{cmp:xbbi})}
\includegraphics[scale=0.75, trim = 0mm 0mm 0mm 5mm,
clip]{chp_singlecarbon/images/nmr/xbbiC}
\vspace{-100pt}
\end{figure}
\end{textblock}
\begin{textblock}{1}(2,2)
\includegraphics[scale=0.8, angle=90]{chp_singlecarbon/images/xbbi}
\end{textblock}
\clearpage
%=-=-=-=-=-=-=-=-=-=-=-=-=-=-=-=-=-=-=-=-=-=-=-=-=-=-=-=-=-=-=-=-=-=-=-=-=-=-=-=-=

%=[xbbj]=-=-=-=-=-=-=-=-=-=-=-=-=-=-=-=-=-=-=-=-=-=-=-=-=-=-=-=-=-=-=-=-=-=-=-=-=-=-=
\begin{textblock}{20}(0,0)
\begin{figure}[htb]
\caption{$^1$H NMR of \CMPxbbj\ (\ref{cmp:xbbj})}
\includegraphics[scale=0.75, trim = 0mm 0mm 0mm 5mm,
clip]{chp_singlecarbon/images/nmr/xbbjH}
\vspace{-100pt}
\end{figure}
\end{textblock}
\begin{textblock}{1}(2,2)
\includegraphics[scale=0.8, angle=90]{chp_singlecarbon/images/xbbj}
\end{textblock}
\clearpage
%%%
\begin{textblock}{20}(0,0)
\begin{figure}[htb]
\caption{$^{13}$C NMR of  \CMPxbbj\ (\ref{cmp:xbbj})}
\includegraphics[scale=0.75, trim = 0mm 0mm 0mm 5mm,
clip]{chp_singlecarbon/images/nmr/xbbjC}
\vspace{-100pt}
\end{figure}
\end{textblock}
\begin{textblock}{1}(2,2)
\includegraphics[scale=0.8, angle=90]{chp_singlecarbon/images/xbbj}
\end{textblock}
\clearpage
%=-=-=-=-=-=-=-=-=-=-=-=-=-=-=-=-=-=-=-=-=-=-=-=-=-=-=-=-=-=-=-=-=-=-=-=-=-=-=-=-=

%% k, l, m, n
%=[xbbk]=-=-=-=-=-=-=-=-=-=-=-=-=-=-=-=-=-=-=-=-=-=-=-=-=-=-=-=-=-=-=-=-=-=-=-=-=-=-=
\begin{textblock}{20}(0,0)
\begin{figure}[htb]
\caption{$^1$H NMR of \CMPxbbk\ (\ref{cmp:xbbk})}
\includegraphics[scale=0.75, trim = 0mm 0mm 0mm 5mm,
clip]{chp_singlecarbon/images/nmr/xbbkH}
\vspace{-100pt}
\end{figure}
\end{textblock}
\begin{textblock}{1}(2,2)
\includegraphics[scale=0.8, angle=90]{chp_singlecarbon/images/xbbk}
\end{textblock}
\clearpage
%%%
\begin{textblock}{20}(0,0)
\begin{figure}[htb]
\caption{$^{13}$C NMR of  \CMPxbbk\ (\ref{cmp:xbbk})}
\includegraphics[scale=0.75, trim = 0mm 0mm 0mm 5mm,
clip]{chp_singlecarbon/images/nmr/xbbkC}
\vspace{-100pt}
\end{figure}
\end{textblock}
\begin{textblock}{1}(2,2)
\includegraphics[scale=0.8, angle=90]{chp_singlecarbon/images/xbbk}
\end{textblock}
\clearpage
%=-=-=-=-=-=-=-=-=-=-=-=-=-=-=-=-=-=-=-=-=-=-=-=-=-=-=-=-=-=-=-=-=-=-=-=-=-=-=-=-=

%=[xbbl]=-=-=-=-=-=-=-=-=-=-=-=-=-=-=-=-=-=-=-=-=-=-=-=-=-=-=-=-=-=-=-=-=-=-=-=-=-=-=
\begin{textblock}{20}(0,0)
\begin{figure}[htb]
\caption{$^1$H NMR of \CMPxbbl\ (\ref{cmp:xbbl})}
\includegraphics[scale=0.75, trim = 0mm 0mm 0mm 5mm,
clip]{chp_singlecarbon/images/nmr/xbblH}
\vspace{-100pt}
\end{figure}
\end{textblock}
\begin{textblock}{1}(2,2)
\includegraphics[scale=0.8, angle=90]{chp_singlecarbon/images/xbbl}
\end{textblock}
\clearpage
%%%
\begin{textblock}{20}(0,0)
\begin{figure}[htb]
\caption{$^{13}$C NMR of  \CMPxbbl\ (\ref{cmp:xbbl})}
\includegraphics[scale=0.75, trim = 0mm 0mm 0mm 5mm,
clip]{chp_singlecarbon/images/nmr/xbblC}
\vspace{-100pt}
\end{figure}
\end{textblock}
\begin{textblock}{1}(2,2)
\includegraphics[scale=0.8, angle=90]{chp_singlecarbon/images/xbbl}
\end{textblock}
\clearpage
%=-=-=-=-=-=-=-=-=-=-=-=-=-=-=-=-=-=-=-=-=-=-=-=-=-=-=-=-=-=-=-=-=-=-=-=-=-=-=-=-=

%=[xbbm]=-=-=-=-=-=-=-=-=-=-=-=-=-=-=-=-=-=-=-=-=-=-=-=-=-=-=-=-=-=-=-=-=-=-=-=-=-=-=
\begin{textblock}{20}(0,0)
\begin{figure}[htb]
\caption{$^1$H NMR of \CMPxbbm\ (\ref{cmp:xbbm})}
\includegraphics[scale=0.75, trim = 0mm 0mm 0mm 5mm,
clip]{chp_singlecarbon/images/nmr/xbbmH}
\vspace{-100pt}
\end{figure}
\end{textblock}
\begin{textblock}{1}(2,2)
\includegraphics[scale=0.8, angle=90]{chp_singlecarbon/images/xbbm}
\end{textblock}
\clearpage
%%%
\begin{textblock}{20}(0,0)
\begin{figure}[htb]
\caption{$^{13}$C NMR of  \CMPxbbm\ (\ref{cmp:xbbm})}
\includegraphics[scale=0.75, trim = 0mm 0mm 0mm 5mm,
clip]{chp_singlecarbon/images/nmr/xbbmC}
\vspace{-100pt}
\end{figure}
\end{textblock}
\begin{textblock}{1}(2,2)
\includegraphics[scale=0.8, angle=90]{chp_singlecarbon/images/xbbm}
\end{textblock}
\clearpage
%=-=-=-=-=-=-=-=-=-=-=-=-=-=-=-=-=-=-=-=-=-=-=-=-=-=-=-=-=-=-=-=-=-=-=-=-=-=-=-=-=

%=[xbbn]=-=-=-=-=-=-=-=-=-=-=-=-=-=-=-=-=-=-=-=-=-=-=-=-=-=-=-=-=-=-=-=-=-=-=-=-=-=-=
\begin{textblock}{20}(0,0)
\begin{figure}[htb]
\caption{$^1$H NMR of \CMPxbbn\ (\ref{cmp:xbbn})}
\includegraphics[scale=0.75, trim = 0mm 0mm 0mm 5mm,
clip]{chp_singlecarbon/images/nmr/xbbnH}
\vspace{-100pt}
\end{figure}
\end{textblock}
\begin{textblock}{1}(2,2)
\includegraphics[scale=0.8, angle=90]{chp_singlecarbon/images/xbbn}
\end{textblock}
\clearpage
%%%
\begin{textblock}{20}(0,0)
\begin{figure}[htb]
\caption{$^{13}$C NMR of  \CMPxbbn\ (\ref{cmp:xbbn})}
\includegraphics[scale=0.75, trim = 0mm 0mm 0mm 5mm,
clip]{chp_singlecarbon/images/nmr/xbbnC}
\vspace{-100pt}
\end{figure}
\end{textblock}
\begin{textblock}{1}(2,2)
\includegraphics[scale=0.8, angle=90]{chp_singlecarbon/images/xbbn}
\end{textblock}
\clearpage
%=-=-=-=-=-=-=-=-=-=-=-=-=-=-=-=-=-=-=-=-=-=-=-=-=-=-=-=-=-=-=-=-=-=-=-=-=-=-=-=-=
			% NMR spectroscopy data